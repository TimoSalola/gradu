\documentclass[12pt,a4paper]{uefrep}
%
% Note, if you are writing in Finnish, modify commented lines in this file and
% modify also report_app.tex if you have appendices.
%
\usepackage[english]{babel}  % If writing in Finnish, replace egnlish with finnish
\usepackage[T1]{fontenc}
\usepackage{cite,url}
\usepackage[dvips]{graphicx}
\usepackage[tbtags]{amsmath}
\usepackage{psfrag}
\usepackage{ae,aecompl}
\usepackage{color}
\usepackage{bookmark}
\usepackage[a-1b]{pdfx}
\usepackage{hyperref}
\usepackage{subcaption}
\usepackage{listings}
\usepackage{xcolor}
%\usepackage{float}
\usepackage{amsthm,amssymb}
\usepackage{tabularx}
\usepackage{longtable} % for table header multicolumn

\usepackage[acronym]{glossaries} % TESTING 27.08.2024
\usepackage[automake]{glossaries-extra}

\definecolor{dkgreen}{rgb}{0,0.6,0}
\definecolor{gray}{rgb}{0.5,0.5,0.5}
\definecolor{mauve}{rgb}{0.58,0,0.82}

\lstset{frame=tb,
  language=Python,
  aboveskip=3mm,
  belowskip=3mm,
  showstringspaces=false,
  columns=flexible,
  basicstyle=\footnotesize,
  numbers=none,
  numberstyle=\tiny\color{gray},
  keywordstyle=\color{blue},
  commentstyle=\color{dkgreen},
  stringstyle=\color{mauve},
  breaklines=true,
  breakatwhitespace=true,
  tabsize=3
}


% LET TESTING



\hypersetup{colorlinks=true,linkcolor=blue,citecolor=red,urlcolor=blue,breaklinks=true}
%
%
\pagestyle{plain}
\bibliographystyle{uefunsrt}
\setlength{\textheight}{20.7cm}
\setlength{\textwidth}{15.0cm}
\setlength{\hoffset}{-2mm}  % Because the location of left margin depends on the printer, adjust this so that the left margin is 32 mm.
%\setlenght{\voffset}{2mm} % If necessary, adjust the top margin to 45 mm with this parameter.
%
\renewcommand{\baselinestretch}{1.2}
\newcommand{\BibTeX}{{\textrm B\kern-.05em{\textsc i\kern-.025em b}\kern-.08em
    T\kern-.1667em\lower.7ex\hbox{E}\kern-.125emX}}
\newcommand{\sep}{;~}

\newenvironment{conditions}
  {\par\vspace{\abovedisplayskip}\noindent\begin{tabular}{>{$}l<{$} @{${}={}$} l}}
  {\end{tabular}\par\vspace{\belowdisplayskip}}


\usepackage{floatrow}
% Table float box with bottom caption, box width adjusted to content
\newfloatcommand{capbtabbox}{table}[][\FBwidth]

\usepackage[font=small,labelfont=bf,tableposition=top]{caption}

\DeclareCaptionLabelFormat{andtable}{#1~#2  \&  \tablename~\thetable}


% ###########################################################

\usepackage{glossaries}

\makeglossaries



\newglossaryentry{albedo}
{
        name=Albedo,
        description={In PV context albedo refers to the fraction of sunlight relfected by a surface. Ground albedo near PV site can be used for estimating how ground reflections contribute to PV system output}
}

\newglossaryentry{inverter}
{
        name=Inverter,
        description={Electronic device used for transforming direct current to alternating current. Often used with PV installations}
}


\newglossaryentry{tilt}
{
        name=Tilt,
        description={One of the two panel angle components. Measured in degrees. Tilt of a solar where the panel normal vector is pointing directly up would be 0$^\circ$ and towards the horizon would be 90$^\circ$}
}

\newglossaryentry{azimuth}
{
        name=Azimuth,
        description={One of the two panel angle components. Measured in degrees, zero point is north and rotation is measured clockwise}
}

\newglossaryentry{nominal capacity}
{
        name=Nominal capacity,
        description={Expected system output when panels are receiving 1000W/m$^2$ and panel temperature is 25$^\circ$C}
}


\newglossaryentry{pv}
{
        name=PV,
        description={Photovoltaic. Refers to solar power technology which uses semiconductors with photovoltaic effect for electricity generation}
}

\newglossaryentry{AOI}
{
        name=AOI,
        description={Angle of incidence. In PV context this is usually the angle between solar panel normal vector and a vector which represents incoming radiation}
}

\newglossaryentry{POA}
{
        name=POA,
        description={Plane of array. In PV context this refers to a 1m$^2$ surface which has equal tilt and azimuth angles as a PV panel installation}
}

\newglossaryentry{DNI}
{
        name=DNI,
        description={Direct Normal Irradiance. Irradiance in Watts per m$^2$ surface with AOI of 0$^\circ$}
}

\newglossaryentry{DHI}
{
        name=DHI,
        description={Diffuse Horizontal Irradiance. Irradiance per 1m$^2$ of shaded surface with tilt of 0$^\circ$. Can be measured with a pyranometer with a solar tracker which blocks direct sunlight}
}

\newglossaryentry{GHI}
{
        name=GHI,
        description={Global Horizontal Irradiance. Irradiance per 1m/$^2$ surface with tilt of 0$^\circ$. Can be measured with a pyranometer}
}

\newglossaryentry{irradiance}
{
        name=Irradiance,
        description={Radiation per unit of surface area, W/$m^2$}
}

\newglossaryentry{solar noon}
{
        name=Solar noon,
        description={In this thesis solar noon is defined as the daily midpoint between sunset and sunrise. This midpoint is easier to measure than the point in time during which Sun crosses the local meridian which is the official definition of the solar noon}
}

\newglossaryentry{cloud free day}
{
        name=Cloud free day,
        description={Day during which the PV system is producing power and clouds do not influence PV production enough for the incluence to be noticeable. Note that clouds may still be present}
}






% ###########################################################

\begin{document}


%% LET TEST

\makeatletter 
\let\c@table\c@figure
\let\c@lstlisting\c@figure
\makeatother


%
\pagenumbering{roman}
%
\include{thesis_front}
\newpage
\include{thesis_abs}


%
\newpage                     % (Un)comment these lines depending on if you have or not have a preface
\phantomsection
\hypertarget{prefacepage}{}
\bookmark[dest=prefacepage]{Preface}
%
\section*{\prefacename}
%Write the preface here. n the preface, you may inform the reader about your motivation to write your thesis and your experiences during the writing of your thesis. You can also use the preface to help the reader get started and to thank people who have helped you with your hesis.
With a strong interest and background in geometry and programming, I have found the real-world problem of panel installation parameter estimation appealing from the beginning. I would like to thank William Wandji and Juha Karhu for their insights on the effects of clouds, snow, and temperature on solar PV installation data. I would also like to thank Luna for being a supportive cat.


\vspace{3mm}
\noindent This research was funded by the Academy of Finland, decision 350695.


\vspace{9mm}
\noindent
Helsinki, the 27th of August 2024 %
%
%%% the signature below has some artistic freedom : )
%
\vspace{11mm}
\hspace{1.6cm}\emph{Timo Salola}

\vfill















     %
%
\phantomsection
\hypertarget{contentpage}{}
\bookmark[dest=contentpage]{Contents}
\tableofcontents
%
\setcounter{page}{0}
\pagenumbering{arabic}
%

\glsaddall

\printglossary[type=main,style=long,nonumberlist]


% ############################## INCLUDES HERE
\include{v2_introduction}
\chapter{Installations and datasets}

\begin{figure}[h]
\centering
\includegraphics[width=0.8\linewidth]{pics/fmikumpula}
\figcaption{FMI Kumpula solar power installation string.}
\label{fig_fmikumpula_panels}
\end{figure}


\noindent The two datasets used in this thesis were provided by FMI. They contain power generation measurements from installations in Kuopio and Helsinki, with the physical parameters listed in table \ref{table_fmi_helsinki_kuopio_parameters}. Due to the high elevation of the installations, shading is unlikely to be a major factor in either of the datasets. The same data was previously used in Herman Böök's \textit{Photovoltaic output modeling}\cite{hbook1} and thus installation parameters and datasets have previously been verified.

The data in the two datasets follows a similar structure as shown in table \ref{table_fmi_kumpula_csv}. This snapshot from the Helsinki dataset shows that the temporal resolution is one measurement per minute and that there are multiple power values for each minute. The fields String 1 and String 2 represent power from two identical sets or strings of solar PV panels installed at the same location and their sum should be near equal to the inverter input. One of these strings is shown in figure \ref{fig_fmikumpula_panels}. Datasets from other sources may differ from the FMI datasets in several ways. Power measurements may be taken at different intervals, once per 1, 5, 15, or 60 minutes, and they are unlikely to contain more than one power value. Due to these reasons, the algorithms in this thesis are designed to operate on datasets with one measurement per minute, and they use only the inverter output value as that is the most likely power value included in solar PV datasets.


\begin{table}[h]

\centering

\begin{tabular}{r|cccc} \hline\hline

Timestamp[UTC] & Inverter out & Inverter in & String 1 & String 2\\ \hline
$2015-08-26$ $03:34$ & $NaN$ & $NaN$ & $0.5$ & $NaN$\\
$2015-08-26$ $03:36$ & $11.1$ & $7.5$ & $2.6$ & $4.9$\\
$2015-08-26$ $03:37$ & $25.4$ & $26.1$ & $9.8$ & $16.3$\\
$2015-08-26$ $03:38$ & $30.7$& $NaN$ & $NaN$ & $0.4$\\
$2015-08-26$ $03:39$ & $46.4$& $44.8$ & $20$ & $24.8$\\
$2015-08-26$ $03:40$ & $3.3$ & $NaN$ & $NaN$ & $0.4$\\
$2015-08-26$ $03:41$ & $29.3$ &  $18$ & $9.1$ & $8.9$\\
$2015-08-26$ $03:42$ & $33.1$& $27.4$ & $10.6$ & $16.9$\\

\vdots & \vdots & \vdots & \vdots & \vdots\\
$2015-08-26$ $12:42$ & $12374.8$ & $14619.1$ & $7152$ & $7467.1$\\
$2015-08-26$ $12:43$ & $15442.2$ & $15482.1 $& $7708.9$ & $7773.2$\\
$2015-08-26$ $12:44$ & $14085.8$ & $12898.7$ & $6387$ & $6511.8$ \\
\vdots & \vdots & \vdots & \vdots & \vdots\\

\hline\hline
\end{tabular}

\tabcaption{A section from FMI's Kumpula solar site PV production data, only the timestamp and inverter output values are used by the algorithms in this thesis. All power measurements are in watts.}
\label{table_fmi_kumpula_csv}
\end{table}





\begin{table}[H]
\centering
\begin{tabular}{r|cc} \hline\hline

 & Helsinki & Kuopio\\ \hline
 Latitude & $60.204^\circ$ & $62.892^\circ$ \\
 Longitude & $24.961^\circ$  &  $27.634^\circ$\\
 Nominal capacity &21 kW & 20.28 kW \\
 Panel tilt & $15^\circ$ & $15^\circ$ \\
 Panel azimuth & $135^\circ$ & $217^\circ$ \\
 Elevation & 17m & 10m\\
\hline\hline
\end{tabular}
\tabcaption{Parameters for the FMI's Kumpula(Helsinki) and Kuopio PV installations as listed in Böök 2020 \cite{hbook1}.}
\label{table_fmi_helsinki_kuopio_parameters}
\end{table}


\begin{figure}[h]
\centering
\includegraphics[width=0.8\linewidth]{pics/irradiancetypes}
\figcaption{Simplified installation diagram for a PV system in ideal conditions.}
\label{fig_simplediagram}
\end{figure}

\noindent
Figure \ref{fig_simplediagram} shows a two dimensional simplification of a solar PV installation. The before unmentioned abbreviation AOI stands for angle of incidence and it is defined as the angle between direct irradiance and solar panel normal. AOI can be calculated with computer programs and software libraries when geolocation, panel tilt and azimuth and the time are known. The azimuth angle is not marked on the figure and azimuth represents the second panel normal component. Azimuth for geographic north is 0 degrees and azimuth is measured in clockwise degrees from north. The angles given for Helsinki($135^\circ$) and Kuopio($217^\circ$) installations tell us that the panels are facing southeast and southwest respectively. 



\section{Visualizing the data}
The figure \ref{fig_oneyear_pointcloud} contains a 3D point cloud generated by plotting one year of data from FMI Helsinki dataset and it shows that there are visible structures in the data. The clerest structure in the 3D -plot is the pattern formed by the first and last non-zero power minutes and this is later used for geolocation estimation. The second pattern is the dome-like shape of the point cloud. This shape can be examined by taking one day slices from the dataset and plotting them individually as shown in \ref{fig_cloudfree_vs_cloudy}. These slices are used for panel installation angle estimation.%The process of installation angle estimation is complicated by weather phenomena and thus a method for filtering out noisy days is needed.

\begin{figure}[h]
\centering
\includegraphics[width=0.8\linewidth]{pics/oneyear2}
\figcaption{One year of data from FMI Kumpula installation as a 3D point cloud.}
\label{fig_oneyear_pointcloud}
\end{figure}

\newpage
\begin{figure}[h!]
\centering
\includegraphics[width=1\linewidth]{pics/cloudfree_vs_cloudy}
\figcaption{Two days from FMI Kumpula dataset with different charasteristics.}
\label{fig_cloudfree_vs_cloudy}
\end{figure}

The presumably cloud free day 2017-123 in \ref{fig_cloudfree_vs_cloudy} has some notable charasteristics in addition to the geolocation derived first and last minutes. The shape resembles a skewed normal distribution and the knee section near 950 minutes is likely caused by the transition from direct sunlight to atmosphere scattered sunlight. This happens when the angle of incidence reaches 90 degrees.

Another measurable trait is the peak power generation minute. Geometric intuition would suggest that this should align with the moment in time with the lowest angle of incidence but this relationship could be fairly complex due to temperature induced efficiency variation and atmopshere induced losses which are higher when the Sun elevation is low.



%Note that this transition appears to be smooth and this may be a result of high reflective losses at high AOI. Similarly, intuition would suggest that the peak power minute occurs when the angle of incidence is at its minimum. Measurable traits such as these could have uses for parameter estimation.
%be used for parameter estimation, but the relationships between figure traits and system parameters can be complicated.


\newpage
\section{Data pre-processing}
The data pre-processing required by the algorithms in this thesis can be split into two gategories, classifying preprocessing and reparing preprocessing. Classifying preprocessing is used to determine if a certain section of data is useful of analysis or not, the primary example here is the cloud free day detection algorithm which is discussed more throroughly in the next chapters. The second type of preprocessing, reparing preprocessing refers to the use of algorithms which fill gaps measurement data or otherwise attempt to repair data which is unusable as is, but which could be used after repairing.

The data preprocessing algorithms used in this thesis load the data from csv files and examine whether individual days in the dataset meet set qualification requirements. These are the minimum and maximum measurement count, whether first and last measurements are taken too close to minute 0 or 1440 and the the percentage of measurements included between the first and last measurement. Figure \ref{fig_accepted_days} contains a comparison on which days in the datasets met the requirements.


\begin{figure}[h]
	
     \centering
     \begin{subfigure}[b]{0.48\textwidth}
         \centering
         \includegraphics[width=\textwidth]{pics/helsinki_accepted_days}
         \caption{Days in Helsinki dataset which met data quality thresholds.}
         \label{fig_helsinki_accepted}
     \end{subfigure}
     \hfill
     \begin{subfigure}[b]{0.48\textwidth}
         \centering
         \includegraphics[width=\textwidth]{pics/kuopio_accepted_days}
         \caption{Days in Kuopio dataset which met data quality thresholds.}
         
         \label{fig_kuopio_accepted}
     \end{subfigure}
     \hfill
     \caption{Requirements were measurement count between 400 and 1200, first minute is 5th or later, last minute is 1435 or earlier. More than 95\% of measurements between first and last minute must be included.}
     \label{fig_accepted_days}
     
\end{figure}


After acceptable days are chosen with the classification algorithm, the next step is data repairing. Missing measurements and Nan values can be linearly interpolated, meaning that if a power measurement or a set of measurements is missing between two datapoints, the missing datapoints are estimated to describe a linear transition breaching the gap between the known datapoints. When noise level is low, linearly approximating the missing values is unlikely to result in signficant errors. After this is done, the resulting data is ready for analysis.



\section{Clear day detection algorithm}
\label{clearskyalgo_chapter}
While the previous preprocessing steps have filtered and repaired days according to measurement counts and data gaps, these algorithms did not filter days based on the amplitude noise present in power measurements. This power noise is often caused by clouds and the resulting generation curves can not be reliably used for curve fitting. The following algorithm steps can be used for automating the process of selecting days with low amounts of high-frequency changes


% If the interference in measurements caused by clouds or other sources is significant, the value of a day for model fitting is reduced. An example of strong interference can be seen in figure \ref{fig_cloudfree_vs_cloudy}. Detecting the presence of such interference with an algorithm would help with automating the process of model fitting as that would eliminate the need to manually select good days from datasets. The following steps describe the process used for cloud free day detection in this thesis.



%\noindent \textbf{Algorithm step by step:}

\begin{enumerate}
  \item Dataset is split into days based on utc timestamps.
  
  \item A copy of the power measurements for each individual day is taken and fed to a low pass filter algorithm.
  
  
  \item A delta value is calculated based on the difference between original power measurements and low pass filtered measurements.

  
  \item Days are discarded if their delta values fail to meet a set threshold. 

  
  
   %If the average difference from step 5 is on average higher than a given treshold value, reject the day.
\end{enumerate}



\noindent The mathematically non-trivial parts here are threshold selection, difference measurement and low-pass filtering. Low-pass filtering is a term borrowed from the field of signal processing, and it refers to any algorithm that removes frequencies higher than a given limit from a signal, allowing lower frequencies to pass. 



Here the filtering is done with discrete Fourier transformations DFT and inverse discrete Fourier transformations IDFT. When a list of numbers is used as the input of DFT, the output is a list of ordered complex numbers, each of which represents a sine wave of a certain frequency, phase and amplitude. The sum of these wave equations forms a continuous approximation of the input values and by sampling the continuous representation, the continuous trigonometric approximation can be transformed back into discrete values. However if the complex numbers are adjusted before the IDFT operation, frequencies can be selectively modified. This means that DFT and IDFT can be used for frequency specific modification of numerical lists, low-pass filtering being one of the possibilities. In this case the low-pass filtering was accomplished by zeroing out complex numbers which do not correspond to the 6 longest frequencies, the resulting smoothening can be seen in figure \ref{fig_cloudfree_algo}.


While this process is somewhat complicated, Fourier transformations are not the only tool for creating low pass filters. Similar results can also be achieved by locally averaging each power value to be the average of nearest $k$ values. Discrete Fourier transformation based methods do however have an advantage in their universality. If the 6 or 7 or $n$ longest frequencies can be determined to be a good low pass filter, then these same frequencies should result in similar outputs no matter the temporal resolution of the power measurement data. Where as a method based on local averages would require a different window size depending on measurement intervals.

The second component is not as complicated as the low pass filtering operation. Measuring the delta between a filtered and unfiltered set of measurements can be done by computing the discrete curve length or as was done here, measuring the absolute average deviation between filtered and unfiltered power measurement. This is shown with the following equations \ref{eq2-1}-\ref{eq2-5}.


\begin{align}
Power &= [p_0, p_1, p_2, \dots , p_n]   \label{eq2-1}\\ 
Power_{filtered} &= [f_0, f_1, f_2, \dots , f_n] \\
Power_{delta} &= [|p_0 - f_0|, |p_1-f_1|, |p_2-f_2|, \dots , |p_n-f_n|] \\
delta_{avg} &= avg(Power_{delta}) \\
delta_{norm} &= delta_{avg}/ max(Power) \label{eq2-5}
\end{align}


The last component is threshold selection. The value $delta_{avg}$ describes the average wattage difference between measured and low pass filtered measured power values. By definition, this delta value is dependent on noise and installation size, limiting its usability. A noise only -delta value can be calculated by normalizing the delta with the $max(Power)$. The resulting $delta_{norm}$ should now be comparable between installations of different sizes. Choosing to reject every day for which $delta_{norm}$ value is higher than 0.05 would eliminate days with higher than 5\% normalized noise.


%By choosing to reject every day for which the $delta_{norm}$ value is higher than 0.05, we can eliminate days where the measured power values and low-pass filtered power values deviate on average more than 5\% from one another,

%from max-normalized power.



%the low pass filtered measured power values. With the Helsinki and Kuopio datasets threshold values as low as 0.5\% still provided some outputs. This threshold value returns more days for the Helsinki dataset, indicating a comparably lower base level of noice in the Helsinki data. This shows that threshold values are installation specific and determined by installation specific noise patterns.


%The thresholds between datasets are not directly comparable. Installation specific shading patterns may increase the base level of noise present in datasets and thus higher thresholds 



%While the The algorithm has some weak points. 



%As the method is based on calculating the delta from a smooth approximation, percentual delta values are unlikely to approach zero unless the modeled data itself is faulty. Depending on the data quality, higher threshold values may need to be used. For example, if the panels are shaded by a tree or some other obstruction during a short period each day, then base level of noise in the measurements would be higher than it is for the FMI datasets. 


%This could be circumvented by 


%As the algorithm is designed for finding the very smoothest of days, the algorithm generates large amounts of false negatives. When this is combined with the interference from sources other than clouds, 


%, the high number of false negatives 


%And as it is designed to seek the very smoothest of days, resulting in a large amount of false negatives. In addition, other forms of interefence such as snow on panels, reflective surfaces near solar panels and clouds which do not block direct sunlight may distract the algorithm. From a human perspective the results of the algorithm may thus be unintuitive as a snowy day may be classified as cloudy even though the interference source was snow on the panels and not the cloud cover. But from the perspective of the processing algorithms, the source of noise in the data is irrelevant.

%Despite these differences


%The terms used in signal processing have mathematical counterparts with some distinctions, for example we can use mathematical structures such as lists, matrices or graphs as signals. And the mathematical near equivalent of high and low frequencies could be defined with the help of delta values between neighboring numerical values in these structures. If the delta values between any two near by values is significant, then this section of the mathematical structure contains high frequency change. Similarly if delta values between any two distant values are high, this is indicative of low frequency change between these two points.






%\noindent There are two mathematically non-trivial components in the algorithm. The first is low pass filtering, a process which treats cloudy and cloud free days differently. This is a concept borrowed from the field of signal processing and the differential treatment can be used to aid in classification. The second non-trivial part is measuring the change between filtered and unfiltered measurements. If the change is minimal, the day can be classified as cloud free.

%The term low pass filtering can be used for any process which takes an input and eliminates frequencies which are higher than a chosen cut off frequency, allowing lower frequencies to pass. With the power measurements, a simple low pass filter could be a running average filter which calculates a new power value as the average of the last $n$ values. The low pass filtering used in this thesis and shown in \ref{fig_cloudfree_algo} was accomplished with discrete Fourier transformations. Discrete Fourier transformations change a list of numbers into a set of sine and cosine waves of different amplitudes, the sum of which forms a continuous representation of the discrete values. This continuous representation can be sampled in order to return the data back into discrete values and by selectively choosing only the waves with longest wavelengths, discrete Fourier transformation can be used for low pass filtering. Good results were achieved by using the 5 to 7 of the longest frequencies generated by the fast Fourier transformation algorithm. 

% n-th degree polynomial approximation of the measurements or new and filtered measurement values could be defined as the average of nearest 20 power measurements.


% The second non-trivial component is measuring the delta between filtered and unfiltered days. This is accomplished by calculating the lenghts of point to point curves in Euclidean space represented by the measurements and filtered measurements by using the equation $\Sigma_{i=1}^{n}  \sqrt[2]{1+ (powers[i]-powers[i-1])^2}$. If the curve lenghts differ by more than x percent, the day can be classified as cloudy.

%The second non-trivial component is measuring the delta between filtered and unfiltered days. This is accomplished by first computing the per minute delta $\delta[i] = |p[i] - p_{lp}[i]|$ where $p[i]$ is the measured power value at index $i$ and $p_{lp}[i]$ is the corresponding power measurement after low pass filter was applied. These delta values can then be used for calculating the average delta per measurement $\delta_{avg} = (\delta[1]+\delta[2] + \dots+ \delta[n])/n$. This normalization is important as without normalization days with lower temporal resolution or shorter day lengths would have lower delta values. The delta value $\delta_{avg}$ can be normalized further as $\delta_n=\delta_{avg}/p_{max}$. This final normalization step results in a delta which represents per measurement deviation as a fraction of max and this final value is comparable between installations of different sizes.

%Final part of cloud free day detection is choosing a threshold value. By manually testing different threshold values, the lowest values which still returns days with the Helsinki and Kuopio datasets are 0.3 and 0.5 respectively. This difference is small but it shows that there is a measurable difference between the datasets and that threshold value should be chosen based on the datasets used. Another issue with the algorithm rises from temporal resolution, if temporal resolution is 1 measurement per 15 minutes and if the reported value is an average of last measurement period, a low pass filter has already been applied. This would not make the algorithm unsuitable for its purpose, but threshold values would have to be adjusted. One method for generalizing the algorithm and avoiding the forementioned issues would be sorting the days based on their smoothness values and then choosing the $n$ smoothest days to operate on. Such modifications may be necessary or beneficial when operating with large amounts of different datasets, but as of now they have not been implemented.


%Most of the program code required for cloud free day detection is included in appendix \ref{cloudfree_code} and the complete source is available on github \cite{github_source}.

%s based on their apparent smoothness values and then operating with the 

%The second non-trivial component is measuring the delta between filtered and unfiltered days. This is accomplished by calculating the lenghts of point to point curves in euclidean space represented by the measurements and filtered measurements by using the equation $\Sigma_{i=1}^{n}  \sqrt[2]{1+ (powers[i]-powers[i-1])^2}$. If the curve lenghts differ by more than x percent


%length of a day by calculating euclidean distance over a day of measurements with the sum function: $\Sigma_{i=1}^{n}  \sqrt[2]{1+ (powers[i]-powers[i-1])^2}$



%Here the algorithm first computes the per measurement error $\delta_i$ for each measurement $p_i$ in the list of powers so that $\delta_i = |p_i - {p_i}_{lp}|$ . This results in a new list of error values is then used in orde to compute the average deviation from 



%Then the average deviation is calculated $\delta_{avg} = (\delta_1+\delta_2 + \dots+ \delta_n)/n$. Finally the delta can be normalized to 



%Here the chosen method is based on computing the length of the discrete curve in minute-power space and the equation used is the sum of point to point distances in Euclidean space. The 



%Alternative approaches such as measuring the total travel on power-axis and calculating error area between filtered and unfiltered curves could also be used. 










%The algorithm splits data into individual days of measurements and the days are evaluated separately. This means that 

%Another important form of data pre-processing is the selection of days which are suitable for model fitting. These so called cloud free or clear days can be distinguished by their smooth power measurement curves as seen in \ref{fig_cloudfree_vs_cloudy}, but the strength of cloud induced noise can be stronger or weaker as well. By making the assumption that clouds induce randomness into the power measurements, we can attempt to measure this randomness and use it as a metric for deciding whether a day is suitable for model fitting or not.

%\noindent To borrow terminology and tools from the field of signal processing, a clear difference between cloudy and cloud free days is the presense of high-frequency noise. Noise means that the alteration to the signal is unwanted and high-frequency signifies that the alteration results in a signal where measurements differ from neighboring measurements. In the case of solar PV power measurements, power $p_t$ for minute $t$ is more likely to differ from the average of the nearest 10 or 100 measurements if the day is cloudy than when the day is cloud free. This difference is visible in \ref{fig_cloudfree_algo} where a cloud free and cloudy day have both been locally averaged.

%Locally averaging or low pass filtering the values can be done in multiple ways. For example, the filtered values could be defined as the average of the nearest 50 values or a polynomial of sufficient degree could be fitted to the measurements data. Regardless of the method used, the important aspect is that the averaging function has to eliminate the apparent randomness from the signal while cloud free days should be hard to distinguish from their unfiltered counterparts. 

%Here the chosen low pass filtering method uses discrete Fourier transform. Fourier transformations are a method of representing number series and functions as sine and cosine waves. In short, the sum of a sine and a cosine wave of frequency $f$ can be used to construct a new sine wave of frequency $f$ with chosen phase. Fourier series combine this property and multiple frequencies in order to construct a frequency representation of the approximated data. As Fourier transformations start from the computation of the longest frequencies, the first $n$ outputs can be used to approximate the general shape of the data.

%This method of filtering may seem complicated as the same results could be achieved by defining new power values as $p_{t_{new}} = avg(p_{t-n}, ...,p_{t-1},  p_t,p_{t+1}, ..., p_{t+n})$ but this would require the turning of the window size parameter $n$ which is dependent on temporal resolution. The benefit of the Fourier series based method is that it is independent from temporal resolution and unlike Taylors polynomial or other localized approximation methods, discrete Fourier series approach does not favor a certain data interval.


\begin{figure}[h]
\centering
\includegraphics[width=0.8\linewidth]{pics/cloudfree_algo}
\figcaption{Cloud free day finder low pass filtering phase.}
\label{fig_cloudfree_algo}
\end{figure}



\newpage






%\noindent \textbf{Note:} that the algorithm listed above is highly dependent on measurement intervals and further tuning could be needed when operating with datasets that have different temporal resolutions. And as is, the algorithm selects days based on their proportionally low high frequency component, thus in theory this algorithm should classify zero power output days as cloud free days. Despite this fault the algorithm seems to work well for the FMI datasets.% as long as the input are selected to contain days close or between the spring and fall equinoxes. 



%The implementation of the clear sky algorithm seems to work well as indicated by \ref{fig-multidaypoavsmeasurements} but there are a few weak points in the algorithm as well. For example if a constant power day is given as the input for the algorithm, the algorithm will classify it as clear sky day even if a constant power day is more likely to be the result of faulty measuring instruments or errors in data preprocessing than a real cloud free day. In addition, the algorithm is unlikely to work well if sections are either removed from the measurements or if there is significant shading affecting the power output of the installation.

%\subsection{Difference between solar days and UTC days}
%For solar power analysis the concept of solar days is fairly useful. Solar days and sun based time measurement systems tend to rely on the angle of the sun and three 


%The timestamps used in solar PV measurements can be assumed to be in UTC +0 time. While this means that timezones or daylight saving time do not have to be accounted for, some operations may become more complicated as well since UTC days and solar days at do not always align. Note that here local solar day is defined as the 

%For example, if a one day slice is taken from a $0^\circ$ longitude installation power generation data, it is rather likely that the solar power generation would occur during an interval which centers around noon or 720 minutes. If the same slice is taken at $90^\circ$ longitude, this generation would be shifted by approximately 360 minutes. This is perfectly normal and expected behavior, but as a result, determining the first and last non-zero minutes of the day can be seen to nontrivial as per figure \ref{fig_poa0vs90}. In the $0^\circ$ plot, the first and last non-zero minutes are approximately 330 and 1100, but should the first and last minutes of the $90^\circ$ plot be defined as 0 and 750, 0 and 1420, -19 and 750 or something else entirely?


%\begin{figure}[h!]
%\centering
%\includegraphics[width=0.9\linewidth]{pics/poa0vs90}
%\figcaption{Approximations of solar power generation at $0^\circ$ and $90^\circ$ longitude.}
%\label{fig_poa0vs90}
%\end{figure}


%\section{Third party datasets}
%Sunny portal etc here

%\section{Required assumptions}The algorithms presented in this thesis work on datasets which 


%Due to the diversity of possible solar PV installations, creating an universal model for parameter estimation is not fesiable. Installations could have been modified during operation, system failures could induce different types of changes in measurement data and 




%\section{Assumptions and possible issues}
%If metadata such as the geographic location or panel installation angles is missing from the datafiles, it is very likely that other critical pieces of information could be left out as well. Were additional modules were installed during operation? Could some panels be installed at different angles? What if the panels are installed in tracking mounts and thus the panel angles vary during each day? These questions are left unaswered and thus some assumptions have to be made. In this thesis we will assume that the panels are installed on fixed mounts, no changes were done during data gathering period and all panels are oriented similarly. We will also assume that there are no major obstacles casting shadows on the panels and that the panels are not self-shadowing, meaning that the panels are not casting shadows on one another.

%Another source of uncertainty is data collection itself. The device responsible for measuring power output values and logging the values has to have a clock for measuring time, but this clock could have be running too slow or fast, resulting in a drifting error in the timestamps. Similarly if the system clock is running at the right speed but it is off by a minute or two, this could cause a bias in the data which would be hard to detect. There is also the question of how measurement timing is done. If the time resolution of the logging device is 15 minutes, is the power value at 12:45 taken during the 45th minute or is the power value the average of the previous 15 minutes as is often done in meteorology? Or could the power value be the average of measurements taken during the interval 12:38 to 12:52? In meoteorology, the last period average would be the standard, but standards may not always be followed.

%is the standard, but standards are not always followed and thus 

%If enough time and effort was spent on algorithm design, in theory it could be possible to detect modifications to PV systems, the presence of variable panel angle systems and clock drifts. But these topics are outside the scope of this thesis and thus the assumption will be made that the 






















\chapter{Solar irradiance simulation tools}
The primary tool used in this thesis for simulating PV generation with different parameters is the python library PVlib. PVlib contains be built functions for estimating solar irradiance, angle of incidence and a multitude of other useful tools. As seen from table \ref{table_poa_simulated_format}, the plane of array(POA) simulations which estimate the irradiance per 1m² of solar panel surface resemble the earlier FMI PV datasets in their structure.

%An example of the irradiance simulation outputs is shown in table \ref{table_poa_simulated_format} and the table shows that the output format is simular to the FMI PV datasets shown earlier in table \ref{table_fmi_kumpula_csv}.




%Having a mathematical model which would simulate the output of a PV system would allow for the parameters of a PV installation to be solved with model fitting. In the best case scenario, we would have a physics based model which would take geographic location, panel installation angles, time of year and panel surface area or power rating as inputs and the output would be similar to the data from FMI Kumpula installation seen in table \ref{table_fmi_kumpula_csv}. Creating a such model is rather challenging as the model has to take into account atmospheric scattering, Sun angles, Sun-Earth distance variation and a multitude of other factors, the consideration of which are far beyond this mathematics thesis. Luckily the modeling of the energy output of solar PV installations has uses for the cost-benefit analysis of solar PV installations and thus pre-existing modeling algorithms are publicly available. 


% TODO REWRITE

%This thesis uses a plane of array irradiance simulation function from the python library PVlib. The function takes geographic location, timestamp and panel angles as inputs. The outputs contain power values which describe the amount of direct and atmospherically scattered light that would hit a square meter sized imaginary solar panel with the input parameters. The sum of these sources is referred as plane of array (POA) irradiance and this value can be used to estimate the output of solar power installations. A section of simulated data is included in table \ref{table_poa_simulated_format}.

%As the model simulates radiation values during clear sky conditions and not the power output of pv installations, the model should be seen as an approximation which is accurate to a certain degree. The differences between the model and recorded measurements could be due to reflectivity of the solar panels, weather conditions, temperature related changes in efficiency, atmospheric composition or a multitude of other factors which the model does not take into account.

\begin{table}[h]

\centering

\begin{tabular}{r|cccc} \hline\hline

Timestamp[UTC] & Minute & POA(W) \\ \hline
$2018-05-30$ $00:00$ &  $0$ & $0.0$\\
$2018-05-30$ $00:01$ &  $1$ & $0.0$\\
$2018-05-30$ $00:02$ &  $2$ & $0.0$\\
\vdots & \vdots & \vdots \\
$2018-05-30$ $ 07:34$ & $454$ & $800.691861$\\
$2018-05-30 $ $07:35$ & $455$ & $802.110516$\\
$2018-05-30 $ $07:36$ & $456$ & $803.517424$\\
\vdots & \vdots & \vdots \\
$2018-05-30$ $ 23:57$ & $1437$ & $0.0$\\
$2018-05-30 $ $23:58$ & $1438$ & $0.0$\\
$2018-05-30 $ $23:59$ & $1439$ & $0.0$\\

\hline\hline
\end{tabular}
\tabcaption{One day of simulated plane of array irradiance values. Note that the minute column is added to the table for convinience and it is reduntant as minutes can be read from the timestamps.}
\label{table_poa_simulated_format}
\end{table}


\subsection{PVlib POA function inputs} 
%PVlib plane of array irradiance can be simulated with 

%\begin{lstlisting}[caption={PVlib POA simulation function header.}, label={poa_header}]
%def get_irradiance(year, day, lattitude, longitude, tilt, azimuth):
%\end{lstlisting}


\noindent The following listing contains the relevant parameters of the plane of array irradiance function and their domains.
%The POA simulation function accepts real --or integer as is the case with the day parameter-- valued parameters in the ranges listed below. 

\begin{itemize}
	\item Year $\in \mathbb{N}$
	\item Day [1, 365/366] $\in \mathbb{N}$
	\item Latitude [-90, 90] $\in \mathbb{R}$%, Finland fits within subrange [59, 70]
	\item Longitude [-180, 180]  $\in \mathbb{R}$%, Finland fits within subrange [19, 32] 
  	\item Tilt [0, 90] $\in \mathbb{R}$
  	\item Azimuth [0, 360[ $\in \mathbb{R}$
\end{itemize}

%\noindent \textbf{Note:} While the function does accept the full latitude and longitude ranges as inputs, it may be beneficial to restrict the range of the coordinate parameters when the approximate location of the installation is known. For example, Finland fits within subrange [19, 32] on the longitude axis and thus it could make sense to restrict the longitude range when examining installations located within Finland.

\vspace{3mm}
\noindent
The day and year parameters can be assumed to be always known as datasets include timestamps and this leaves four unknown system parameters. These four parameters span a 4D parameter space of possible PV installations. The size of this parameter space is connected to the difficulty of the parameter estimation problem. As the parameters are in $\mathbb{R}$, the amount of sensible combinations is the product of the discretization of each of the parameter ranges. Tilt and azimuth parameters can be discretized as integers, resulting in $90*360$ or $32400$ unique angle space points. The geographic latitude and longitude coordinates are somewhat harder to discretize. In an arbitrary 0.1 degree geographic discretization there would be $1800*3600$ or 6480000 coordinate combinations. The product of these two sums results in $2*10^{11}$ unique parameter combinations.

A parameter space this large is difficult to examine exhaustively. Even a system capable of evaluating 100 000 angle space points per second would need approximately 23 days in order to evaluate all of the possible combinations. However if the parameters can be solved in isolation from oneanother, the required computational time would decrease significantly. Instead of $90*360*1800*36000$, there would be $90+360+1800+3600$ combinations to evaluate. This highlights how important it is to break problems into smaller pieces whenever possible and much of this thesis focuses on how this can be done with solar PV parameter estimation. 


\newpage
\section{PVlib POA evaluation}

Before implementing the parameter estimation functions, the POA simulations should be tested against known measurement data. Figure \ref{fig-multidaypoavsmeasurements} displays that in clear sky conditions the pvlib irradiance model is following real world measurements closely with a few exceptions. And during cloudy days the measurements are often lower than the clear sky model would indicate, but they can also peak higher than they would during cloud free days. These increases of power generation above clear sky estimated power are likely to be caused by the additional sunlight reflected from clouds towards solar panels in partly cloudy weather conditions and it shows that cloud induced noise can be positive as well as negative. 



\begin{figure}[h]
\centering
\includegraphics[width=0.9\linewidth]{pics/multiday_vs_neat}
\figcaption{Power output of FMI Kumpula PV installation and the pvlib POA simulation computed with the parameters \ref{table_fmi_helsinki_kuopio_parameters}. Horizontal axis on the graphs corresponds to time and vertical axis marks the estimated power values. The purpose of the graphs is to display the different shapes and deviations from POA models and thus axis names and numbers were left out. Upper row contains randomly selected days while as the lower row has days chosen by a clear sky algorithm mentioned in chapter \ref{clearskyalgo_chapter}. Measurements are from 2017. POA irradiance values were multiplied by 20 in order to match the curves values on power axis.}
\label{fig-multidaypoavsmeasurements}
\end{figure}



% On the 70th and 150th day, the first and last minutes do not seem to be exactly the same as in the simulation, but the difference seems minor and it is occuring in different directions.






%\textit{The following claims are unverified conjectures, but the smooth shape and the early date of the first graph could hint that the increased peak production on the 70th day could be due to reflections from snow, while as the more irregular production on the 150th and 240th day would seem to indicate that the variation is caused by clouds. Filtering out days such as the 150th or the 240th from the dataset should be rather simple as the high frequency component is noticeable, but low frequency deviations such as the smooth increase in production of the 70th day could prove to be more difficult to detect algorithmicly.}






\newpage
%\section{Influence of different parameters on the PVlib poa model}
%\label{influence_parameters}



\begin{figure}[ht!]
\centering
\includegraphics[width=0.9\linewidth]{pics/poa_eval_new_crop}
\figcaption{Influence of changes in PVlib simulation parameters on generated power output curves. Control shows FMI Helsinki measurements and simulation with the same parameters as the Helsinki installation. Simulated power values are multiplied by 19 in order to match values on y-axis.}
\label{fig_poa_different_parameters}
\end{figure}

\noindent By varying the different simulation parameters as shown in figure \ref{fig_poa_different_parameters}, we can examine the relationships between parameters and power generation curves. This can help us understand if there are usable patterns in the data. In the best case scenario each of the simulation function inputs would affect one measureable property in the irradiance plots and their relationship would be bijective. To give an example, if the peak power minute was isolated from all other parameters than the longitude and the relationship between longitude and peak power minute was linear, it would be possible to solve the peak power minute to longitude function with just a few plane of array irradiance simulations.

In the exact opposite case where every measureable property of irradiance plots is affected by every input parameter, solving the parameters would be much harder or even impossible. For example if all of the parameters influenced the same traits to different extents and the system was not bijective, multiple parameter combinations could result in the same simulated power graph. In a such system there would not be a single solution but rather a set of possible solutions.

The problem of solving installation parameters would appear to be somewhere in between the two extremes. The longitude parameter would seem to shift the curve along the time axis where as tilt and azimuth parameters do not affect the first or last non-zero minutes but they do affect the shape of the curve. Observations of parameter to trait interactions are listed on table \ref{table_traits}.



\begin{table}[H]
\centering
\begin{tabular}{r|cc} \hline\hline

 Parameter & Traits affected\\ \hline
 Latitude & Shape, first and last minute times\\
 Longitude & First and last minute times\\
 Tilt & Shape\\
 Azimuth & Shape\\

\hline\hline
\end{tabular}
\tabcaption{Function input to observed trait table.}
\label{table_traits}
\end{table}


%Base on these observations, the relationship between longitude and the First and last minute times would seem like the best starting point for parameter solving.

%If the POA model is assumed to be accurate, the model could be used to simulate the effects of different parameters on power generation. This could provide insights into the relationship between patterns in the data and the parameters of the system. The relevant parameters to simulate and their default values can be seen in table \ref{table_default_parameters_poa_simulations}. In the following simulations, only one of the default parameters is varied. This is done in order to isolate the effect of individual parameters.



%\begin{table}[!ht]
%\centering
%\begin{tabular}{r|c} \hline\hline

% Parameter & Value \\ \hline
% Day & $180$  \\
% Latitude & $60^\circ$  \\
% Longitude & $28^\circ$  \\
% Panel tilt & $30^\circ$ \\
% Panel angle & $180^\circ$  \\
%\hline\hline
%\end{tabular}
%\tabcaption{Default parameters for POA simulation used in this section. }
%\label{table_default_parameters_poa_simulations}
%\end{table}


\newpage

\subsection{Influence of different longitudes}
\label{section_different_longitudes}

\begin{figure}[ht!]
\centering
\includegraphics[width=1\linewidth]{pics/poa_var_lon}
\figcaption{First and last non-zero minutes of each day from year long simulations at different longitudes.}
\label{fig-poa_var_lon2}
\end{figure}

\noindent Based on earlier observations listed in table \ref{table_traits}, solving the longitude of installations would seem like a sensible starting point. The figure comparing the effects of different parameters seemed to suggest that the relationship between longitude and significant minute times is very close to linear and the same is seen here in figure \ref{fig-poa_var_lon2}. In Hagdadi 2017 \cite{navid_australian_article} and in Williams 2012 \cite{older_solar_solver_article} this relationship was used in order to determine the geographic longitude. The algorithms used by both of the articles relies on calculating an approximation for the time of the solar noon based on the average of the first and last minutes, this solar noon minute is then translated into a geographic longitude coordinate.





%There are at least two ways of estimating the longitude from the UTC solar noon time. First method is based on fitting a linear equation to a list of known solar noon to longitude-pairs. This would result in an equation of the form $f(x) = 0.25^\circ* x + b$ where the solar noon minute $x$ is multiplied by the constant $0.25$. The constant of $0.25^\circ$ comes from dividing a full circle by the amount of minutes in a day, 1440. The constant $b$ is around $-180^\circ$ and it is the result of solar noon occuring close to noon.
%In the figure \ref{fig-poa_var_lon2}, the relationship between the first and last minutes of a day and the geographic longitude can be seen to be linear. This linear equation should be of the form $f(x) = 0.25^\circ* x + b$ where the solar noon minute $x$ is multiplied by the constant $0.25$. The constant of$0.25^\circ$ comes from dividing a full circle by the amount of minutes in a day, 1440. The constant $b$ is roughly $-180$ degrees as that is the offset required for adjusting solar noon from



%and the figure \ref{fig-poa_var_lon2} it would seem that the relationship between longitudes and first and last minutes is a good starting point for parameter so


%at least very close to linear. In Hagdadi 2017 \cite{navid_australian_article} and in Williams 2012 \cite{older_solar_solver_article} this relationship was used in order to determine the geographic longitude. The algorithms used by both of the articles relies on calculating an approximation for the time of the solar noon based on the average of the first and last minutes, this solar noon minute is then translated into a geographic longitude coordinate.

% and a similar algorithm is detailed in []..




\newpage

\subsection{Influence of different latitudes}
\label{section_different_latitudes}

\begin{figure}[ht!]
\centering
\includegraphics[width=1\linewidth]{pics/poa_var_lat}
\figcaption{First and last non-zero power minutes of each day from year long simulations at different latitudes}
\label{fig_poa_var_lat}
\end{figure}

\noindent The latitude simulations show that the day length stays fairly consistent for locations close to the equator, but with latitudes of $50^\circ$ and higher, the day to day variation is significant. These POA simulations would imply that the region around equinoxes is ideal for day length based analysis as there day length is always well defined and the rate of change can be measured. 


\newpage

\section{Increasing the accuracy of solar PV simulations}
\label{section_increased_accuracy_simulations}


\begin{figure}[h]
\centering
\includegraphics[width=0.7\linewidth]{pics/poa_eval_single_day}
\figcaption{Figure comparing Pvlib simulated POA irradiance and actual measured PV output of FMI Helsinki installation.}
\label{fig-poa_eval_single_day}
\end{figure}

\noindent Figure \ref{fig-poa_eval_single_day} compares the simulated and measured PV output during a sunny day in Helsinki. The shapes of the curves match quite closely, but there are some discrepancies. The estimated power values for the peak production hours are higher than measured power values and a similar phenomena is seen near minute 1000. These discrepancies can be explained by real world phenomena as PVlib POA values describe the amount of radiation reaching the surface of a solar panel. This is different from the power output of a PV system as thermal losses and panel reflections are not taken into account. This section examines an improved PV model which includes these losses.

\newpage

\subsection{Improved PV model}

\begin{figure}[h]
\centering
\includegraphics[width=1.0\linewidth]{pics/uml2}
%\captionsetup{labelformat=empty}
\figcaption{Original POA-based PV model and the improved model with reflection and temperature estimation.}
\label{fig-pv_model}
\end{figure}


\noindent 
The improved PV model introduces some new concepts, among which are the following irradiance types.

\begin{itemize}
\item DNI is diffused normal irradiance which represents direct sunlight received by a 1m² -sized plane with AOI of 0 degrees. Note that athmosphere scattered or ground reflected irradiance are not components of DNI.

\item DHI is diffuse horizontal irradiance which represents the irradiance reaching a shaded 1m² -sized plane with tilt of 0 degrees. DHI represents atmosphere scattered irradiance.

\item GHI is global horizontal irradiance and it represents the amount of irradiance per 1m² -sized plane with tilt of 0 degrees. GHI is made up of direct irradiance and atmosphere scattered irradiance. GHI combined with albedo can be used to estimate the ground reflected irradiance.
\end{itemize}

\noindent
These irradiance types are present in both the simpler PVlib POA based model and the improved PV model. The difference between the two models is that in the POA-model, PVlib internally calculates three irradiance components DNI, DHI and GHI and projects them to the panel surface, returning this panel projected irradiance as plane of array irradiance. Where as in the more complex PV model, PVlib estimates DNI, DHI and GHI, each of which are processed by multiple functions before they are combined into an estimated power output value, resulting in a physically more accurate PV output estimation.

\subsection{Panel surface projections}

The first step in the improved model is panel surface projection. Sandia National Laboratories, the original author of PVlib suggests the following two equations for DNI and GHI projection and five alternative models for DHI projection. Perez 1990 model \cite{perez} was chosen for DHI projections due to the use of the model by a co-worker at FMI and inclusion in Sandia suggested projection models \cite{sandia_poa_dhi}. For implementation of DNI, DHI and GHI equations, see thesis source code.

%As there are three irradiance types with different apparent sources and thus directions, three irradiance projection functions are needed. The simplest way to project DNI and DHI to panel surface is to use the equations $DNI_{proj}= DNI* sin(AOI)$ and $DHI_{proj}= DHI* sin(tilt)$ both of which are mathematically trivial. However ground reflected irradiance projection is not as simple. The original author of PVlib, Sandia National Laboratories recommends the use the following two equations for DNI and GHI, offering five different models for DHI. 


\noindent\textbf{DNI projection}\cite{sandia_poa_dni}
%
\begin{equation}
\begin{split}
\label{sandia_eq_dni}
DNI_{proj}(DNI, AOI)= DNI*cos(AOI)
\end{split}
\end{equation}
Where $AOI$ is the angle of incidence.


\noindent\textbf{GHI projection}\cite{sandia_poa_ghi}
%
\begin{equation}
\begin{split}
\label{sandia_eq_ghi}
GHI_{proj}(GHI, albedo, tilt)= GHI*albedo*\frac{1-cos(tilt)}{2}
\end{split}
\end{equation}

\noindent Where $albedo = 0.151$. This represents ground reflectivity near the solar PV installation. The value varies significantly depending on the area, season and time of day. A fixed value is used as terrain, weather and snow cover are assumed to be unknown.

\noindent $tilt$ is the tilt angle of the PV installation, this is $15^\circ$ for both FMI installations.

%\noindent\textbf{DHI projection}\cite{sandia_poa_dhi}
%\begin{equation}
%\begin{split}
%\label{sandia_eq_dni}
%GHI_{proj}(DHI, tilt) = See source
%\end{split}
%\end{equation}

\newpage
\subsection{Reflection estimation and absorbed irradiance}
The following Solar irradiance reflection equations originate from 2001 Martin and Ruiz paper \cite{solar_reflections}. This study used a physical contraption in which solar panels were rotated under a light beam from a solar simulator. Values from the reflective losses equations represent the fraction of irradiance reflected away from the panels e.q. a $DNI_{reflected}$ value of 0.24 would tell us that 24\% of irradiance is lost due to reflections and 76\% is absorbed. 

\noindent\textbf{DNI reflective lossess}

% THIS IS F_A(Beta) in original source
\begin{equation} 
\begin{split}
\label{solar_reflecive_dni_loss} 
DNI_{reflected}= exp(-\frac{1}{a_r}(c_1 p_1 +c_2 p_1^2))
\end{split}
\end{equation}


% THIS IS F_D(Beta) in original source
\noindent\textbf{DHI reflective lossess}
%
\begin{equation}
\begin{split}
\label{solar_reflecive_dhi_loss}
DHI_{reflected}= exp(-\frac{1}{a_r}(c_1 p_2 +c_2 p_2^2))
\end{split}
\end{equation}

\noindent\textbf{GHI reflective lossess}
%
\begin{equation}
\begin{split}
\label{solar_reflecive_ghi_loss}
GHI_{reflected} = \frac{exp(-cos(\alpha)/a_r)- exp(-1/a_r)}{1-exp(-1/\alpha_r)}
\end{split}
\end{equation}

\noindent Where

$a_r = 0.159$ Empirical reflectance constant for polycrystalline silicon solar PV panels.

$c_1= \frac{4}{3\pi}$  Fitting parameter 1.

$c_2 = -0.074$ Fitting parameter 2.

$p_1= sin(tilt) + \frac{tilt-sin(tilt)}{1-cos(tilt)}$

$p_2= sin(tilt) + \frac{\pi - tilt-sin(tilt)}{1+cos(tilt)}$


\noindent\textbf{Absorbed irradiance}
\begin{equation}
\begin{split}
\label{solar_reflecive_absorbed}
POA_{absorbed} = DNI(1-DNI_{r})+ DHI(1-DHI_{r})+GHI(1-GHI_{r})
\end{split}
\end{equation}

\noindent Where $DNI$, $DHI$ and $GHI$ are solar irradiance values in watts and $DNI_r$, $DHI_r$, $GHI_r$ refer to \ref{solar_reflecive_dni_loss}, \ref{solar_reflecive_dhi_loss} and \ref{solar_reflecive_ghi_loss}.





\subsection{Panel temperature estimation}
Panel temperatures need to be estimated as the efficiency of solar panels depends on panel temperature. This can be accomplished with the following King et al 2004 model\cite{king2004} if air temperature, wind speed and absorbed radiation are known. Note that as the model does not take the thermal capacity of the panels into account, the actual panel temperature is likely smoother and delayed compared to modeled temperature.

\noindent\textbf{Panel temperature}
\begin{equation}
\begin{split}
\label{panel_temp}
T_{panel} = T_{air} + POA_{absorbed} e ^{C_a+ C_b*w}
\end{split}
\end{equation}

\noindent Where 

$T_{air}$ is air temperature in $^\circ C$.

$C_a = -3.47$ Model fitting constant.

$C_b = -0.0594$ Model fitting constant. 

$w$ is wind speed at panel elevation.


\vspace{6mm}

\noindent As air temperature and wind speed are unknown, dummy values have to be used for both variables. Air temperatures of 20$^\circ C$ and wind speed of 2m/s were used as dummy values.

\newpage
\subsection{Output estimation}
Now that the absorbed radiation and panel temperature are known, the output can be estimated with the following Huld et al 2010 model\cite{huld2010}.

\noindent\textbf{PV output model}
\begin{equation}
\begin{split}
\label{pv_output_model}
P_{output} = P_{rated}  P_n  \eta_{rel}
\end{split}
\end{equation}

\noindent Where 

$P_{rated}$ is the rated power of the system in kW.

$P_n = \frac{POA_{absorbed}}{1000}$

%$POA_{absorbed}$ is \ref{solar_reflecive_absorbed}.

$\eta_{rel}= 1+k_1 ln(P_n) + k_2 ln(P_n)^2 + T_{diff}k_3 + k_4 ln(P_n) + k_5 ln(P_n)^2 + k_6 T_{diff}^2$

In $\eta_{rel}$ the variables $k_1$ to $k_6$ are fitting parameters and $T_{diff}$ is $T_{panel}-25^\circ C$.

\begin{table}[h]
\begin{tabular}{l|l|l|l|l|l}
k1        & k2        & k3        & k4       & k5       & k6       \\ \hline
-0.017162 & -0.040289 & -0.004681 & 0.000148 & 0.000169 & 0.000005
\end{tabular}
\end{table}

\noindent %This leaves the rated power of the system as an unknown parameter. 
For unknown PV systems, the unknown $P_{rated}$ value can be estimated by comparing the measured power output to simulated power output with a theoretical 1kW system. Visualization of this process with POA simulations and real measurements is shown in \ref{fig_area_match}.


\begin{figure}[h]
\centering
\includegraphics[width=0.7\linewidth]{pics/areamatching2}
\figcaption{Measurement data from FMI Helsinki dataset, POA irradiance simulation was computed with FMI Kumpula coordinates and installation angle parameters.}
\label{fig_area_match}
\end{figure}

\newpage
\subsection{Model improvement results}
The models can be compared with the original measurement data with the following equation.



\noindent\textbf{Model error}
\begin{equation}
\begin{split}
\label{pv_model_delta}
Delta_{model} = \sum_{t=1}^{1440} |P_{measured}(t) - P_{simulated}(t)| /1440
\end{split}
\end{equation}

\noindent The delta value represents a normalized per minute deviation between the model and the meausred data. Normalization by division with 1440 results in low delta values as this assumes that the system is constantly generating power regardless of the time of day and thus the resulting delta value is somewhat missleading but for comparison between models these values are still useful.


Delta values for detected cloud free days for FMI Helsinki installation are on average 244 for the POA model and 145 for the improved model based on a sample of 22 cloud free days. For FMI Kuopio the POA model achieved a delta of 487 and improved model 320 with 42 day dataset. This would indicate that the improved PV model is a better approximation for clear sky PV output than the PVlib POA model.


\begin{figure}[h]
\centering
\includegraphics[width=0.99\linewidth]{pics/pvlibsimplecomplex}
\figcaption{Comparison of PVlib POA and the improved PV power generation model on a day from Helsinki dataset.}
\label{fig-poa_eval_simplecomplex}
\end{figure}

\noindent Figure \ref{fig-poa_eval_simplecomplex} shows what the improvement looks like in a best case scenario. The model can be seen to perform better during peak production hours and during the last productive hours. This model would appear to be a more accurate approximation for PV generation than the POA model, but the POA model still has uses in applications where faster computation is beneficial or where only the timing of the first and last non-zero minutes matter. Which of these models is being used will be mentioned in each use case in the following chapters.


%that the improved PV model performs better than the simple POA-model at 1000 minutes where the AOI is high and thus reflective losses from direct irradiance are significant. And during peak production minutes where the panel temperatures are likely high enough to cause thermal losses.

%The improved model would appear to be a more accurate approximation for PV generation than the POA model, but the POA model still has uses in applications where faster computation is beneficial or where only the timing of the first and last non-zero minutes matter. Which of these models is being used will be mentioned in the following chapters.

%The model performance could be further tuned as inverter efficiency and other possible factors were not taken into account. However as there are only two datasets, this would increase the likelyhood of overfitting and thus the improved model is left as is.


%is near 90$^\circ$ and during peak production hours where increased temperature effects performance. This is promising but the complexcity of the model also increases the likelyhood of overfitting. Nevertheless this improved model with 2m/s wind speed, 8m elevation and 25 $^\circ$C ambient temperature and the POA model will both be used in the following chapters depending on the charasteristics of parameter estimation functions.

%\noindent Figure \ref{fig-poa_eval_single_day} show that simulations are fairly accurate but there seems to be some deviations between the simulated values and the measurements. The two significant deviations are the noise during peak power generation period and the smooth decrease in power generation during the last hours of the day. The peak power generation noise is likely to be caused by decreases in efficiency due to heat the occasional cooling from gusts of wind. The second deviation between the model and actual measurements is likely to be caused by panel reflections as the south-east orientation of the solar panels results in a high angle of incidence during the last hours of the day.

%Modeling these physical phenomena is possible to an extent if the components of plane of array irradiance are used instead of the POA values. As per Sandia National Laboratories \cite{sandia_poa}, the three components of plane of array irradiance are direct normal irradiance(DNI), global horizontal irradiance(GHI) and diffuse horizontal irradiance DHI. A physically accurate model would compute the absorbed radiation by first projecting the irradiance components to the plane of array adn the estimating the losses caused by reflections. After this is accomplished, the absorbed irradiance could be used to estimate panel temperature and power output.









\chapter{Estimating geographic location}
\label{chapter_est_geoloc}
In order to evaluate the performance of longitude and latitude estimation functions, it may prove useful to be able to translate the error values from degrees to kilometers. The following Equations \ref{equation_latitude_delta_km} and \ref{equation_longitude_delta_km} can be used to approximate the deltas of longitude and latitude estimation functions in kilometers. Note that these functions are only approximations as they rely on the assumption that the Earth is a perfect sphere and not an irregular ellipsoid.


\hfill \break
%%%%%%%%%%%%%%%%%%%%%%%%%%%%%%%%%%%%%%%%%%%%%%%%%%%%%%%%%%%%%%%%%%%%% START
\noindent\textbf{Latitudinal distance to kilometers}(Distance on North-South axis)
%
\begin{equation}
\begin{split}
\label{equation_latitude_delta_km}
Distance_{latitudinal}(lat\_d)=(40 000km/360^\circ)* lat\_d
\end{split}
\end{equation}

\noindent Where 

$lat\_d$ is the distance between two points in degrees latitude.

40000km is an approximation for Earths circumference.

\vspace{5mm} %5mm vertical space
%%%%%%%%%%%%%%%%%%%%%%%%%%%%%%%%%%%%%%%%%%%%%%%%%%%%%%%%%%%%%%%%%%%%% END

%%%%%%%%%%%%%%%%%%%%%%%%%%%%%%%%%%%%%%%%%%%%%%%%%%%%%%%%%%%%%%%%%%%%% START
\noindent\textbf{Longitudinal distance to kilometers at given latitude}(Distance on East-West axis)
%
\begin{equation}
\begin{split}
\label{equation_longitude_delta_km}
Distance_{longitudinal}(lon\_d, lat)=(40 000km/360^\circ)* cos(lat)*lon\_d
\end{split}
\end{equation}

\noindent Where 

$lon\_d$ is the distance in degrees longitude.

$lat$ is the latitude for which the distance is calculated.



\vspace{5mm} %5mm vertical space

\noindent As long as the deviations are small enough and highly accurate error values are not needed, the total error in absolute terms can be estimated by using the latitudinal and longitudinal distances as the x and y coordinates on a cartesian plane and computing the euclidean distance between the origin and resulting point.

%%%%%%%%%%%%%%%%%%%%%%%%%%%%%%%%%%%%%%%%%%%%%%%%%%%%%%%%%%%%%%%%%%%%% END

\newpage
\section{Estimating geographic longitude}
\noindent As mentioned in Sections \ref{section_different_latitudes} and \ref{section_different_longitudes}, the geographic location of a PV system has a strong correlation to the timing of the first and last non-zero measurements of each day whereas the influence of tilt and facing parameters appears to be nonexistent. The relationship would seem to be so clear that without further analysis it would be tempting to use fairly simplistic mathematical models for these estimations. The following longitude estimation Function \ref{equation_naive_longitude} can be derived with two assumptions. These assumptions are that solar noon occurs at 12:00 or 720 minutes at longitude 0$^\circ$ each day and at 6:00 or 360 minutes at 90$^\circ$. Rest of the values can then be linearly interpolated. Note that here solar noon refers to the midpoint between the first and last non-zero minute which is different from astronomical solar noon which occurs nearly at the same time.

As only the first and last non-zero minute times are relevant for longitude and latitude estimation, PVlib POA model is used for both longitude and latitude estimation.

\hfill \break


%The simpler of these relationships is the relationship between the longitude and first and last non-zero minute times. Based on the figure \ref{fig-poa_var_lon2}, this relationship seems to be very close to linear. This makes the use of a simple linear equation rather compelling. 



%%%%%%%%%%%%%%%%%%%%%%%%%%%%%%%%%%%%%%%%%%%%%%%%%%%%%%%%%%%%%%%%%%%%% START

\noindent\textbf{Naive solar noon to longitude equation}
%
\begin{equation}
\begin{split}
\label{equation_naive_longitude}
Longitude(sn)=180^\circ-\frac{360^\circ}{1440}*sn
\end{split}
\end{equation}
Where $sn$ is the approximated solar noon minute calculated by taking the average of first and last non-zero power generation minute of a day. % $0.25$ or $360/1440$ corresponds to the longitude degrees per minute and $180$ is used as offset value. This naive equation assumes that solar noon occurs at 12:00 or 720 minutes each day at $0^\circ$ longitude. Note that here solar noon refers to the midpoint between the first and last non-zero minute which is different from astronomical solar noon which occurs nearly at the same time.

\hfill \break
%%%%%%%%%%%%%%%%%%%%%%%%%%%%%%%%%%%%%%%%%%%%%%%%%%%%%%%%%%%%%%%%%%%%% END


\noindent The simplicity of Equation \ref{equation_naive_longitude} makes the equation appealing, but the assumption of solar noon occuring at 720 minutes should still be verified. In Figure \ref{fig_solarnoons} solar noons can be seen to occur at around 720 minutes at longitude 0$^\circ$ but they can also be observed occuring 15 minutes earlier or later than that. The cause for this pattern is a combination of Earth's axial tilt and elliptical orbit \cite{eot}. This 15 minute delta would translate into an error range of $(\pm 15/1440)*360^\circ = \pm 3.75^\circ$ degrees or approximately $\pm 200 km$ at the latitudes of Helsinki according to the Equation \ref{equation_latitude_delta_km}.


Knowing that the PV installation is within a 400 kilometer wide slice may in some cases be enough for determining the country in which the PV installation is located in, but for most other purposes this level of accuracy is unlikely to be valuable. Fortunately the naive model can be improved upon by taking the solar noon timing variation into account.

\hfill \break

\begin{figure}[ht!]
\centering
\includegraphics[width=1\linewidth]{pics/solarnoons2}
\figcaption{Approximations of solar noon minutes based on PVlib POA function at longitude $0^\circ$ for year 2023.}
\label{fig_solarnoons}
\end{figure}



\newpage
%%%%%%%%%%%%%%%%%%%%%%%%%%%%%%%%%%%%%%%%%%%%%%%%%%%%%%%%%%%%%%%%%%%%% START
\noindent\textbf{Improved longitude estimation function }
%
\begin{equation}
\begin{split}
\label{equation_longitude_estimation_2}
Longitude(sn)= \frac{360}{1440}(sn_{poa}-sn)
\end{split}
\end{equation}

\noindent Where $sn$ is the solar noon estimate based on measurement data and $sn_{poa}$ is the simulated solar noon at $0^\circ$ longitude. The new function parameter $sn_{poa}$ compensates for the variation seen in Figure \ref{fig_solarnoons}.


%%%%%%%%%%%%%%%%%%%%%%%%%%%%%%%%%%%%%%%%%%%%%%%%%%%%%%%%%%%%%%%%%%%%% END

\vspace{5mm}


\noindent
%The improved function \ref{equation_longitude_estimation_2} should be much more accurate than the earlier longitude function. as the errors in estimates should primarily be the result of bias in the measurements data.


%In figure \ref{fig_solarnoons_poa_vs_measurement} the solar noon estimates calculated by taking the average of first and last non-zero minute of the day can be seen to preceed the solar noons by roughly 8 minutes. This systematic bias could be explained by the east facing orientation of the panels and correcting for the bias could could be fairly difficult. 

\vspace{0.5cm}
%The theoretical accuracy at this stage could be as high as $\frac{360^\circ}{1440} = 0.25^\circ$ but 
\noindent The improved algorithm should no longer have a systematic error of up to 15 minutes after the difference between solar time and UTC time variation has been taken into account. In addition to correcting for the irregular solar noon timing, the algorithm can be improved even further by using it on larger sections of data and averaging the results, or alternatively, the algorithm could be applied only on selected cloud-free days where the expected errors are likely to be smaller.

If the unfiltered multi-day approach is used, choosing the right day range is crucial. If the range is too narrow, a single outlier value can distort the results significantly, however if the whole year is used, certain periods of the year may contain more noise than others and thus their use could decrease the accuracy of the results. The two scatterplots in Figure \ref{fig_first_last_kuopio_helsinki} show that the data quality from the very first and last days of the year seem to be significantly worse than the data from the longest days of the year. Based on these visualizations, days inside the range 100th to 280th would seem best suited for first and last minute sensitive analysis algorithms for both Helsinki and Kuopio installations.



%noise present in non-zero minutes is higher in the Kuopio installation data. 

%noise present in power generation measurements may result in significant errors in estimates based on individual days. These errors can be mitigated by using the algorithm on larger sections of data and averaging the results, or alternatively the algorithm could be applied only on selected cloud free days where the expected errors are likely to be smaller. If the unfiltered multi-day approach is used, choosing the right day range is crucial. If the range is too narrow, a single outlier value can distort the results significantly, however if the whole year is used, certain periods of the year may contain more noise than others and thus their use could decrease the accuracy of the results. The two scatterplots in figure \ref{fig_first_last_kuopio_helsinki} show that the data quality from the very first and last days of the year seem to be significantly worse than the data from the longest days of the year. The same graph would also seem to suggest that overall data quality decreases the further north the installation is. Based on these visualizations, days outside the range of 100th to 280th would seem unsuitable for first and last minute based analysis between latitudes $60^\circ N$ and $63^\circ N$.


\begin{figure}[ht!]
\centering
\includegraphics[width=0.9\linewidth]{pics/first_last_helsinki_kuopio2}
\figcaption{First and last non-zero power minutes of each day during years 2017 to 2021 from FMI Helsinki and Kuopio datsets.}
\label{fig_first_last_kuopio_helsinki}
\end{figure}

\clearpage

\subsection{Longitude estimation results}
The longitude estimation algorithm was tested on a day range of 125th to 250th for each year in the FMI Helsinki dataset and the results are seen in figure \ref{fig_longitude_estimation_helsinki}. By taking the mean of the longitude predictions for each year in the dataset, the resulting longitude estimates varied between -0.804$^\circ$ to 0.0766$^\circ$ from known installation longitudes. Results for the Kuopio dataset shown in figure \ref{fig_longitude_estimation_kuopio} were similar with mean delta range of -0.1042$^\circ$ to 0.9424$^\circ$.


In kilometers the year with the highest mean deviation of -0.804$^\circ$ in the Helsinki dataset corresponds to an error of approximately 45 kilometers as per \ref{equation_longitude_delta_km}. With Kuopio results the highest delta of 0.9424$^\circ$ is 49 kilometers respectively.

\begin{figure}[ht!]
\centering
\includegraphics[width=0.9\linewidth]{pics/longitude_v2_helsinki}
\figcaption{Box plot of longitude estimation using FMI Helsinki dataset. Boxes mark 25\% to 75\% quantile intervals and the orange middle lines are the medians. Correct longitude marked as a black line near 25 degrees.}
\label{fig_longitude_estimation_helsinki}
\end{figure}


\begin{figure}[ht!]
\centering
\includegraphics[width=0.9\linewidth]{pics/longitude_v2_kuopio}
\figcaption{Box plot of longitude estimation using FMI Kuopio dataset. Boxes mark 25\% to 75\% quantile intervals and the orange middle lines are the medians. Correct longitude marked as a black line near 25 degrees.}
\label{fig_longitude_estimation_kuopio}
\end{figure}


%The improved algorihm was tested on the day range of 125th to 250th of each year from both FMI datasets and the results can be seen on the Table \ref{table_geolocator_results}. For the Helsinki installations these estimates are all off by less than $0.3^\circ$ while as the Kuopio installation deltas are a bit higher with max of just over $1^\circ$. More impressively, the mean delta of multiple years for the Helsinki dataset is just $0.07^\circ$ and $0.46^\circ$ for Kuopio. In kilometers, the mean deltas can be approximated to 4 and 28 kilometers respectively. The lower accuracy of the Kuopio estimations could be due to multitude of factors ranging from differences in local climate or lower elevation of the installation among others. 




\newpage
\subsection{Possible issues and further development ideas}
PV output measuring systems may not be able to accurately measure when the power output is zero. The capacitance of electrical circuits, instrument accuracy, or other phenomena could lead to small positive or negative power readings even when the PV system is not producing any electricity. Small negative power output readings occur in the FMI datasets and the first and last non-zero power minutes are used instead of first and last greater than zero power minutes, this results in a bias of -3$^\circ$ in the longitude predictions. If similar positive false readings are present in datasets, this could complicate the estimation of the times when power production begins or stops.


While experimenting with the solar minute estimation functions, a curious trait was found. In Figure \ref{fig_solarnoontimes}, the average of the first and last minute is approximately the same for each day at different latitudes as long as the latitude is below 50$^\circ$. As the latitude is increased, the solar noon estimates begin to deviate significantly, becoming strongly skewed after 70$^\circ$. 

At first this behavior seems strange as astronomical solar noon should occur at the same time when longitude and the day are the same regardless of latitude. However as the solar noon estimates are calculated based on the first and last non-zero irradiance minute of the day, it would make sense that the estimations could be off by significant amount during equinoxes due to rapid changes in day lengths. Thus the day range was split into two sections and the color of days was chosen according to their distance from equinoxes. Line color was chosen based on distance from equinoxes where darker lines were the furthest away. This structure visible in the figure suggest that day length variation is a likely to be a factor in the phenomena. 


In theory, if the same bias occurs in both the measurements and the modeled first and last positive power minutes, no corrections would be needed. The effect should be also lessened by using longer day ranges for predicting longitudes or by making sure that the intervals include an equal amount of days from both halves of the year as the direction in which this error occurs appears to be the opposite on both halves of the year.

Improvements in the algorithm accuracy could be achieved via by increasing the sampling interval of the irradiance simulations. PVlib POA simulations include a parameter for sampling frequency which is currently set to 1-per-minute in order to match the measuring frequency of FMI datasets. This could be increased to 1-per-second and the added resolution could help in determining more accurate estimates for solar noon times, resulting in possible gains in algorithm accuracy.

PVlib POA model was used instead of the more complex reflection and temperature aware model. This could be done as the geolocation is connected to first and last non-zero minute times which should be the same for both models. However even the POA model might be overly complicated as only two time values are needed. The usage of a much simpler model could increase computational speed significantly.




\begin{figure}[]
\centering
\includegraphics[width=1\linewidth]{pics/solarnoontimes2}
\figcaption{Each line represents a set of solar noon estimates based on first and last simulated non-zero power minute at different latitudes. Days are split based on whether they belong to the first or second half of the year. Darker line color marks days which are further away from equinoxes.}
\label{fig_solarnoontimes}
\end{figure}







%\textbf{Longitude estimation notes}

\newpage 
\section{Estimating geographic latitude}
Similarly to the longitude, the latitude of an installation is strongly connected to the timing of the first and last non-zero minutes of the day. This means that PVlib POA simulations can be used instead of the more complex PV model.

In earlier Figure \ref{fig_poa_var_lat}, the simulated first and last minutes can be seen to change day by day at varying rates based on the latitude. In mathematical terms it could be said that the slope of the day-to-first-minute function is determined by the latitude of the installation. And for the days around equinoxes, and at higher latitudes of $50^\circ$ to $70^\circ$, this relationship would seem to be bijective as per earlier Figure \ref{fig_poa_var_lat}. The following algorithm is based on the former observations.


\hfill


\subsection{Latitude algorithm}
%\noindent\textbf{Latitude algorithm}
\begin{enumerate}
  \item Simulate first non-zero minutes for a day range A to B at a given latitude.
  
  \item Fit a line to the simulated day to first minute pairs from step 1. Save the line slopes and the corresponding latitude.
  
  \item Repeat steps 1 and 2 for multiple latitudes.
  
  \item Create a slope to latitude graph from the earlier steps.
  
  \item Fit an n-degree polynomial equation to the graph from step 4.
  
  \item Select the day range A to B from a solar PV dataset and fit a line to the first non-zero power minutes. Feed the slope of this line to the polynomial from step 5. Value of this polynomial is an approximation of the latitude of the PV system.
  
 
  
\end{enumerate}

\noindent
\textbf{Notes:}
%A visualization of the algorithm can be seen in the following figure \ref{fig_slope_to_latitude}. 
Due to seasonal differences in data quality, polar winters and the midnight sun, the range of days chosen for the algorithm is important. If the range is short, individual outliers in measurements can result in large errors. Whereas if the range is too long, it will be harder to choose the range while avoiding low data quality sections. In the algorithm visualization Figure \ref{fig_slope_to_latitude}, the range of 250th to 300th seems to result in a good linear fit.
\newpage

%\subsection{Latitude algorithm visualization}
\begin{figure}[ht!]
\centering
\includegraphics[width=1\linewidth]{pics/slope_to_latitude3}
\figcaption{Left shows the linear relationship between day and simulated first minutes. Right shows the relationship between the slope angle and latitude.}



%the latitude algorithm steps 1 and 2 for day range 250 to 300 at latitude 60 and the linear model fitting. Second graph shows steps 3 and 4 in which slope angles are plotted agains latitudes. The fitted 3rd degree polynomial is $f(x)= 0.305x^3 - 4.607x^2 + 25.953x + 19.908$.}
\label{fig_slope_to_latitude}
\end{figure}



\begin{table}[!ht]
\centering
\begin{tabular}{r|c|c} \hline\hline
 \multicolumn{3}{ c }{FMI Kumpula[250-300]}\\\hline
Year & Predicted latitude & Error\\
2021 & $61.365^\circ$ &  $1.161^\circ$\\
2020 & $64.493^\circ$ &  $4.289^\circ$\\
2019 & $63.121^\circ$ & $2.917^\circ$\\
2018 & $61.190^\circ$ & $0.986^\circ$\\
2017 & $57.515^\circ$ & $-2.789^\circ$\\


\hline\hline
\end{tabular}
\tabcaption{Results from estimating the latitude of FMI Kumpula PV installation with the preceeding algorithm. Day range of 250th to 300th was used.}
\label{table_geolocator_latitude_results}
\end{table}
\vspace{3mm}
\noindent

\subsection{Improving latitude prediction algorithm}
The results of the algorithm shown in Table \ref{table_geolocator_latitude_results} are somewhere in the correct region, but the delta of over $4^\circ$ in the 2020 estimate is significant and much higher than the error of the longitude estimation algorithm. The first step in improving the algorithm would be the use of last non-zero minutes as well as the first non-zero minutes. This doubles the amount of outputs from the algorithm and while doubling the amount of outputs does not directly increase the accuracy of the algorithm, it can provide additional insights into the performance of the algorithm. %This is especially valuable as the available datasets are small.



\begin{table}[!ht]
\centering
\begin{tabular}{r|c|c|c|c} \hline\hline

\multicolumn{5}{ c }{FMI Kumpula[250-300]}\\\hline
Year & First min. p. & Error &  Last min. p. & Error \\
2021 & $61.365^\circ$ &  $1.161^\circ$ & $63.685^\circ$ & $3.481^\circ$\\
2020 & $64.493^\circ$ &  $4.289^\circ$ & $64.288^\circ$ & $4.084^\circ$\\
2019 & $63.121^\circ$ & $2.917^\circ$ & $66.762^\circ$ & $6.558^\circ$\\
2018 & $61.190^\circ$ & $0.986^\circ$ & $60.230^\circ$ & $0.026^\circ$\\
2017 & $57.515^\circ$ & $-2.789^\circ$  & $62.256^\circ$ & $2.052^\circ$\\

\hline\hline
\end{tabular}
\tabcaption{Latitude algorithm with added output for last minutes based prediction.}
\label{table_geolocator_latitude_results_f_and_l}
\end{table}


%Figure \ref{table_geolocator_latitude_results_f_and_l} shows that predictions based on first and last minutes contain similar errors. This is to be expected and it shows that both first and last minutes of each day have similar potetial for latitude estimation. 


\noindent The second step in improving the algorithm is choosing the best possible day range for latitude estimation. One way of choosing the day ranges would be by testing multiple day ranges and choosing the range which results in the lowest average absolute error from the known latitude, but this is problematic as the correct latitude should not be assumed to be known. However if there were multiple datasets with complete metadata, this could be used in order to find universally well-behaving day ranges.

\textit{Standard deviation minimization} is another option for day range selection. As there are two estimated latitude values per year, datasets with $n$ years of data would provide $n*2$ estimated latitude values. Standard deviation of these values could expected to be small if the day interval does not contain days with bad data quality and this means that the interval selection can be automated. Following Figure \ref{fig_heatmap3d2} shows a heatmap for FMI Helsinki latitude estimation. Lowest found standard deviation occured with the day interval 180th to 310th with standard deviation of 0.3788$^\circ$.


%However exhaustively searching the day interval space with reasonable intervals can be slow and as seen in the figure \ref{fig_heatmap3d2}, low standard deviation intervals are common in the parameter space. As a result, an arbitrarily chosen long interval is likely to perform almost as well as the very best algoritmicly chosen interval.

%This same figure indicates that an arbitrarily chosen long interval is likely to perform almost as well as the very best algoritmicly chosen interval.


%If the day range is selected well, these estimations could be expected to be tightly grouped. This tight grouping can be measured by calculating the standard deviation and thus the day interval with the lowest standard deviation could be expected to result in the best latitude estimation. A such method relies on the assumption that low standard deviation correlates with good preditions, but this assumption can be shown to be false or missleading. The following figure \ref{fig_heatmap3d2} shows that 

\begin{figure}[]
\centering
\includegraphics[width=0.8\linewidth]{pics/std_dev_helsinki_latitude}
\figcaption{Standard deviation map for FMI Helsinki latitude estimation algorithm. Tightest grouping of latitude results was achieved with day interval 180th to 310th.}
\label{fig_heatmap3d2}
\end{figure}

\newpage

\subsection{Latitude estimation results}
Latitude estimation appears to be more challenging to optimize than longitude estimation. With the Helsinki dataset the found low standard deviation interval 180-310 predictions shown in Table \ref{table_geolocator_latitude_results_180_310} do not seem to be better than those achieved by an arbitrarily chosen interval 190-250 shown in Table \ref{table_geolocator_latitude_results_190_250}. This observation is supported by the flatness of the standard deviation space in Figure \ref{fig_heatmap3d2} which also suggest that large sections in the day interval space have a similarly low standard deviation. Results for the Kuopio installation have higher variance but the scatter appears to otherwise be similar as seen on Table \ref{table_geolocator_latitude_results_kuopio}.

%Achieved results are fairly good and they will be left as is.


%Results for the Kuopio dataset contain similar variance with the interval 190-280 resulting in predictions shown in Table \ref{table_geolocator_latitude_results_kuopio}.



\begin{table}[!ht]
\centering
\begin{tabular}{r|c|c|c|c} \hline\hline

\multicolumn{5}{ c }{FMI Helsinki latitude estimation results(180-310)}\\\hline
Year & First min. p. & Error &  Last min. p. & Error \\

2021 & $60.7453^\circ$ &  $0.5409^\circ$ & $61.107^\circ$ & $ 0.9026^\circ$\\
2020 & $60.4969^\circ$ &  $0.2925^\circ$ & $60.82^\circ$ & $0.6156^\circ$\\
2019 & $60.5981^\circ$ & $0.3937^\circ$ & $60.4222^\circ$ & $0.2178^\circ$\\
2018 & $60.6537^\circ$ & $0.4493^\circ$ & $60.5745^\circ$ & $0.3701^\circ$\\
2017 & $61.416^\circ$ & $1.2116^\circ$  & $61.2206^\circ$ & $1.0162^\circ$\\

\hline\hline
\end{tabular}
\tabcaption{Estimated latitudes for FMI Helsinki Kumpula dataset with day range of 180th to 310th}
\label{table_geolocator_latitude_results_180_310}
\end{table}

%The following two tables contain examples of the results of the latitude estimation algorithm. Results of the latitude estimation algorithm are not as good as the longitude estimations, but for now they will suffice. The predictions follow a similar patter as the previous longitude estimations in that predictions for the Helsinki installation are grouped tighter and their errors are lower than those of the Kuopio installation. 

\begin{table}[!ht]
\centering
\begin{tabular}{r|c|c|c|c} \hline\hline

\multicolumn{5}{ c }{FMI Helsinki latitude estimation results(190-250)}\\\hline
Year & First min. p. & Error &  Last min. p. & Error \\

2021 & $59.792^\circ$ &  $-0.677^\circ$ & $60.186^\circ$ & $-0.334^\circ$\\
2020 & $59.792^\circ$ &  $-0.412^\circ$ & $60.186^\circ$ & $-0.018^\circ$\\
2019 & $59.896^\circ$ & $-0.308^\circ$ & $59.558^\circ$ & $-0.646^\circ$\\
2018 & $59.945^\circ$ & $-0.259^\circ$ & $59.463^\circ$ & $-0.741^\circ$\\
2017 & $60.577^\circ$ & $0.373^\circ$  & $60.008^\circ$ & $-0.196^\circ$\\

\hline\hline
\end{tabular}
\tabcaption{Estimated latitudes for FMI Helsinki Kumpula dataset with day range of 190th to 250th}
\label{table_geolocator_latitude_results_190_250}
\end{table}

\noindent 

\begin{table}[!ht]
\centering
\begin{tabular}{r|c|c|c|c} \hline\hline

\multicolumn{5}{ c }{FMI Kuopio latitude estimation results(190-280)}\\\hline
Year & First min. p. & Error &  Last min. p. & Error \\
2021 & $62.626^\circ$ &  $-0.266^\circ$ & $63.197^\circ$ & $0.305^\circ$\\
2020 & $62.259^\circ$ &  $-0.633^\circ$ & $61.895^\circ$ & $-0.997^\circ$\\
2019 & $62.983^\circ$ & $0.091^\circ$ & $62.708^\circ$ & $-0.184^\circ$\\
2018 & $62.722^\circ$ & $-0.170^\circ$ & $62.874^\circ$ & $-0.018^\circ$\\
2017 & $61.669^\circ$ & $-1.223^\circ$  & $61.152^\circ$ & $-1.740^\circ$\\

\hline\hline
\end{tabular}
\tabcaption{Estimated latitudes for FMI Kuopio Kumpula dataset with day range of 190th to 280th.}
\label{table_geolocator_latitude_results_kuopio}
\end{table}

\clearpage

\subsection{Possible issues and further development ideas}
PVlib POA based first and last minute estimations are slower to compute than necessary as only two timestamps are needed. The use of a simpler sunrise and sunset equation would increase the speed significantly, allowing for the use of brute force day range selection algorithms.

Different methods could also be used. In Hagdadi 2017 \cite{navid_australian_article} latitude estimations are done by fitting solar irradiance models with 3 unknown parameters to power generation measurement data. The latitude deltas of 1.65$^\circ$ to 3.42$^\circ$ in the 2017 article are higher than those achieved in this thesis, however as the datasets, geographical regions and algorithms are different, direct comparison can not be made.

%by this paper, however as the datasets, geographical regions and algorithms are different, direct comparison can not be made.

In earlier Figure \ref{fig_slope_to_latitude} the slope to latitude fitting is not perfect. The use of higher degree polynomials may result in a closer fit which in turn may result in a marginal improvement in latitude prediction accuracy. Alternatively a piecewise linear interpolation based fitting could be used. Both methods can in theory result in higher estimation accuracy at the risk of overfitting.
 

The two day intervals 180 to 310 and 190 to 250 may be biased. For the interval 180 to 310 all of the intervals result in positive error values whereas with 190 to 250 these errors mostly negative. This suggests that there could be biases in the data which do not influence standard deviation of the results. This will be left as a possible topic of study for future studies.



\newpage
\section{Combined latitude and longitude estimations}
As it is unlikely that the longitude and latitude estimation algorithms are used in isolation from one another, their results should be examined together. This can be done by plotting the estimated locations on a map. Here the two installations in Helsinki and Kuopio and their predicted locations per year are plotted side by side. Plotted latitudes are from earlier tables \ref{table_geolocator_latitude_results_190_250}, \ref{table_geolocator_latitude_results_kuopio} and longitudes are from the longitude prediction algorithm using an arbitrarily chosen day range of 190th to 280th.

\begin{figure}[h]
	
     \centering
     \begin{subfigure}[b]{0.45\textwidth}
         \centering
         \includegraphics[width=\textwidth]{pics/geolocationmap2}
         \caption{FMI Helsinki.}
         \label{fig_geolocationhelsinki}
     \end{subfigure}
     \hfill
     \begin{subfigure}[b]{0.45\textwidth}
         \centering
         \includegraphics[width=\textwidth]{pics/geolocationmap3}
         \caption{FMI Kuopio.}
         
         \label{fig_geolocationkuopio}
     \end{subfigure}
     \hfill
     \caption{Geolocation estimations for FMI datasets.}
     \label{fig_geolocationmap}
\end{figure}

\noindent In the Helsinki predictions figure, the estimated geolocations are scattered around the known installation location, showing very little bias and some random noise. Similar behavior can be seen in Kuopio predictions where two outliers 2017 and 2020 deviate more significatly. One degree on the latidue axis is approximately 110 km regardless of latitude and longitude, one degree of longitude is 56km at $60^\circ$ N and 50 km at $63^\circ$ N. As the variance is strongest on the latitude axis, it is likely that the latitude prediction algorithm is more sensitive to variations in the data and further development should be focused on more accurate latitude prediction and day range selection.


The results can be compared to the data resolution. One minute delta in measurements corresponds to a longtudinal shift of 14 kilometers in longitudinal axis at $60^\circ$. As the point cloud width is approximately $1^\circ$ or 50km, the estimates can be thought to have a longitudinal range of 50km, $1^\circ$ or 3.5 minutes with nearly the same values for the Kuopio installation. As the temporal resolution of measurements is 1 minute, the algorithm should not be limited by temporal resolution.
\chapter{Estimating panel angles}
Solar panel installation angles are a large factor in deciding the energy output of a PV system. If panel angles can be freely chosen during planning and installation phases, it can make sense to either optimize for total power generation or power generation during peak consumption hours. This means that even if installation angles could be freely chosen, installation angles are unlikely to be the same for every system in the same geographical region. Panel angles may also be restricted by installation sites and mounting types.

% Panel racks have to be installed based on the available area 

%Installation type also plays a factor in choosing the panel angles. Panel angles may be restricted by 

 %In so called flush rooftop installations, the panels are installed to run along the roof and so the angles can not be freely chosen. Panels can also be rack mounted and these racks tend to be installed and oriented based on the available area as is the case with \ref{fig_fmikumpula_panels}. Due to the forementioned reasons, panel angles vary from installation to installation.

One reason for lacking of faulty metadata is that panel angles can be difficult to measure accurately. The tilt angle of the panels or the angle between the panel normal and zenith(the point directly above) can easily be measured with an angle ruler and a bubble level, but the azimuth angle of the panels is much harder to measure with the same degree of accuracy. If an accurate compass is used and the difference between the magnetic north and the geographic north is taken into account, metal structures and electrical systems nearby can still distort local magnetic fields enough to cause errors in measurements. The challenges in taking accurate measurements are not insurmountable, but they may contribute to the inaccuracies and the lack of available information in PV installation parameter metadata. 

The space of possible panel installation angles can be thought as a half unit sphere in a spherical coordinate system where each point on the surface represents a direction to which the normal of the solar panels could be directed towards. A visualization of parameter space in 3D and 2D is shown in \ref{fig_halfdome} and \ref{fig_anglespace1}. The 3 dots in the subfigure \ref{fig_anglespace1} mark the zenit for which azimuth is not well defined(red), the installation angles of FMI Helsinki installation azimuth $135^\circ$ tilt $15^\circ$(blue) and a close to power generation maximized installation with directly south facing panels with the tilt of $45^\circ$(black).


\begin{figure}[h]
	
     \centering
     \begin{subfigure}[b]{0.35\textwidth}
         \centering
         \includegraphics[width=\textwidth]{pics/halfdome}
         \caption{The space of possible angles as a 3D half sphere surface. Each point represents a possible tilt and azimuth combination.}
         \label{fig_halfdome}
     \end{subfigure}
     \hfill
     \begin{subfigure}[b]{0.35\textwidth}
         \centering
         \includegraphics[width=\textwidth]{pics/polarplot}
         \caption{2D projection of the angle space, distance from center denotes the tilt angle of the panels and angle marks the azimuth.}
         
         \label{fig_anglespace1}
     \end{subfigure}
     \hfill
     \caption{Angle space visualizations.}
     \label{fig_anglespace}
\end{figure}


\noindent Estimating panel installation angles requires the use of multiple functions, each of which can be defined in multiple ways. These functions are defined in the following sections.
\begin{itemize}
  \item Prediction error function for quantifying how good a prediction was when the correct panel parameters are known.
  \item Model fitness function for measuring the difference between simulated power values and measured power values.
  %\item Multiplier matching function for matching the magnitude of simulated power values with the magnitude of measurements.
  \item Angle space discretization function for discretizing the angle space into $n$ discrete points which can then be tested with model fitness function.
\end{itemize}




\section{Prediction error function}
%The first part in developing a panel installation estimation algorithm is creating a metric for measuring how well the algorithm performs. 
In this thesis, the proposed error estimation method combines the tilt and azimuth delta values into one error angle value, the angular distance between two points on a spherical surface. The goal is then to develop a panel angle estimation function which achieves the lowest angle error value with the available datasets.

Alternative approaches can also be chosen as the function or functions for measuring the distance between two points in angle space can be defined in multiple ways. The simplest way is to use the delta of known tilt and azimuth angles as two separate error values without normalizing in any way. This method was used in Hagdadi's 2017\cite{navid_australian_article} article but such values are not direcly comparable between installations as the significance of azimuth delta depends on tilt angle.

%Measuring the distance between two coordinate pairs in angle space is more complicated than measuring errors in latitudinal or longitudinal degrees. This difficulty rises from how the azimuth angle lines converge at the pole, resulting in a coordinate system where azimuth delta values are disconnected from the phenomena which they are trying to describe. If the tilt angle is near zero, azimuth delta becomes meaningless but at high tilt angles, even small azimuth delta values can be significant. If this is not corrected for, using azimuth and tilt deltas (changes in angles) as algorithm performance metrics would incorrectly suggest that the data quality of low tilt installations is lower than that of high tilt installations, or that the system is less capable of estimating the parameters of low tilt installations. Due to these reasons, a better way of measuring the distance between two points is needed, luckily the center angle between two points on an unit sphere is easy to solve with geometry and the resulting equation is rather simple.

%While there are no issues with using angles to denote the direction of the panels, the angle values do not map the possible panel angles into the angle space in a way which would make measuring the difference between two angle space coordinates easy. The issues rises from how the azimuth angle lines converge at the pole, resulting in a coordinate system where azimuth delta values are disconnected from the phenomena which they are trying to describe. For example, a 45 degree azimuth delta is fairly significant at tilt of 90 degrees but almost insignificant at 15 degree tilt. If this is not corrected for, using azimuth and tilt deltas (changes in angles) as algorithm performance metrics would incorrectly suggest that the data quality of low tilt installations is lower than that of high tilt installations, or that the system is less capable of estimating the parameters of low tilt installations. Due to these reasons, a better way of measuring the distance between two points is needed, luckily the center angle between two points on an unit sphere is easy to solve with some geometry and the resulting equation is rather simple.

\vspace{3mm}
\noindent\textbf{Deriving angle space distance equation}

\noindent Let $v= [v_1, v_2]$ and $k = [k_1, k_2]$ be two component angle-space vectors so that $v_1$, $k_1$ $\in$ $[0,90]$ and $v_2$, $k_2$ $\in [0,360]$. These vectors represent points on the surface of a unit sphere and their components are the angles of spherical coordinate system. The cartesian coordinates of these points are:
	\begin{align}
	x_v &= sin(v_1)cos(v_2)\\
	y_v &= sin(v_1)sin(v_2)\\
	z_v &= cos(v_1)
  \end{align}
  And
  \begin{align}
	x_k &= sin(v_1)cos(v_2)\\
	y_k &= sin(v_1)sin(v_2)\\
	z_k &= cos(v_1)
  \end{align}
\noindent And the cartesian distance between these two points can be calculated with the following equation:
\begin{align}
	d = \sqrt{(x_v-x_k)^2 + (y_v-y_k)^2+(z_v-z_k)^2}
\end{align}


\noindent The two points and the origin form an isoceles triangle with the sides from the origin to the vector end points having the length of 1 while the distance between the vector end points is the same as d.

\noindent As the lengths of three sides are known, the angles of the triangle can be calculated with the cosine rule. 
\begin{align}
	a^2 &= b^2 + c^2 - 2bc \cos(A)
\end{align}
Where
\begin{conditions}
 a     &  Side opposing the angle A, same as earlier value d \\
 b     &  Side opposing angle B, value is 1  \\   
 c	   &  Side opposing angle C, value is 1
\end{conditions}
\noindent Substituting known values into the cosine equation.

\begin{align}
	a^2 &= b^2 + c^2 - 2bc \cos(A)\\
	d^2 &= 1^2 + 1^2 - 2 \cos(A) \\
	d^2 &= 2 - 2 \cos(A)
\end{align}

\noindent Solving for angle A
\begin{align}
	d^2 &= 2-2\cos(A)\\
	2 \cos(A) &= 2 -d^2 \\
	\cos(A) &= \frac{2-d^2}{2} \\
	A &= \cos^{-1}(\frac{2-d^2}{2})
\end{align}

\noindent Renaming $A$ as $Error$.

\begin{align}
	Error &= \cos^{-1}(\frac{2-d^2}{2}) \label{errorangle}
\end{align}


\noindent By first calculating the distance between the vectors using equations 5.1-5.7 and then substituting the distance into equation 5.16, the resulting angle can then be used as an error value between two panel angle measurements. Python code based on this proof is included in appedix \ref{angular_distace_appendix}.


% This error value is the same as the angle between two points on the surface of a sphere. %In addition, if the deviation occurs only on the tilt axis, the error value and the tilt error are the same.% This makes the error values intuitive.
%In some ways, this method of calculating an error angle is analogous to moving the two angle vectors so that one of them aligns with the 0 tilt point and computing the tilt delta between the two points. Because of this, the error values are fairly intuitive and the error value should better represent the actual difference between installations than other error values calculated via other means.


\vspace{5mm}

%\section{Angle estimation error functions}
%The process of angle estimation requires the use of error estimation functions. The first function is needed for testing the accuracy of the algorithm by translating the known installation angles and the estimated angles into a meaningful error value. This is nontrivial as moving by a set angle value on the tilt and azimuth axis in spherical coordinate space result in different cartesian distances depending on the starting point. 


%The second function or set of functions is needed for evaluating how well a simulated irradiance curve fits the measurement data. Functions of this type can be used for parameter estimation by testing out possible parameter combinations and and choosing the combination which results in the lowest error between the simulated and measured values.


\newpage
\section{Simulation fitness function}
\label{section_simulation_error_function}

The earlier model error function defined in \ref{pv_model_delta} can be re-used as the simulation fitness function. By then computing multiple simulations with different panel angles and choosing the simulation with the best fitness, meaning lowest delta value, the panel angle values can be estimated.

\vspace{6mm}
\noindent\textbf{Simulation fitness function}
\begin{equation}
\begin{split}
\label{pv_model_delta}
Delta_{model} = \sum_{t=1}^{1440} |P_{measured}(t) - P_{simulated}(t)| /1440
\end{split}
\end{equation}

\noindent As the function normalizes delta values by division with 1440, the delta for shorter days is lower than the delta for comparable long days. This should not matter for parameter estimation as the algorithm works by minimizing delta for each day independently. 

Visualization of the function is shown in \ref{fig_simulation_fitness} where simulation with tilt of 90 degrees and azimuth of 135 is tested against known power measurements from a day in FMI Helsinki dataset, resulting in an average delta of 2959.5 W delta per minute.



\begin{figure}[h]
\centering
\includegraphics[width=0.8\linewidth]{pics/halfdome} % WAS pics/measured_vs_simulated
\figcaption{Visualization of fitness function.}
\label{fig_simulation_fitness}
\end{figure}




%The earlier simulation error function defined in \ref{pv_model_delta} can be used for evaluating how well a simulated power output curve fits measured power data. By then simulating the power output with several installation parameters and 


%By simulating power generation with different parameter combinations and choosing the set of parameters with the lowest delta value, 



%Simulation error function measures how much the predicted power generation values vary from the measured power generation values. The purpose of the simulation error function is to be able to generate a single numerical value which describes how well a certain parameter combination models the measurements. By then testing out multiple parameter combinations, the combination with the lowest simulation error function value should be the best fit and the parameters used for the simulation should be within a small error of the physical parameters of the solar PV installation. In \ref{fig_error} the error between a cloud free day and randomly chosen set of wrong simulation parameters is visualized.


\newpage
\section{Angle space discretization}\label{angle_space_discretization}
The next step is angle space discretization. The panel angles are denoted with a doublet of tilt and azimuth values, ranging from 0 to 90 and 0 to 360 respectively. If the tilt and azimuth axes are discretized individually in steps of 5 so that tilt is [0, 5, 10, 15... 90] and azimuth [0, 5, 10, 15... 355], the permuations of these tilt and azimuth values create an even grid in the euclidean projection of angle space where x = tilt, y = azimuth. However as the physical phenomena represented by the angle values is not a point on a flat plane but a point on a half-sphere surface, this results in an uneven discretization seen in figure \ref{fig_5step}. 

Sphere discretization problems are relevant for 3D graphics and real world problems involving geometry and so there are pre-existing methods available for discretization. One of the mathematically more elegant methods is the Fibonacci lattice which was used in a similar fashion in González 2009 \cite{Gonzlez}. The mathematical formulation of similar lattices is an older process and an earlier example is found in Vogel 1979\cite{fibolat_old}. The following mathematical notation for the lattice is based on a code sample included in a blog post by Vagner Seibert \cite{medium_fibolat_equation}.




\noindent \textbf{Fibonacci lattice point n of k equation}
\begin{align}
	s &= n + 0.5 \\
	\phi &= acos(1 - 2 s / k) \\
	\theta &= \pi s (1 + \sqrt{5})
\end{align}
Where $n$ is the point number, $k$ is the amount of points, $\phi$ is the panel tilt angle and $\theta$ is the azimuth angle.
\begin{align}
	x &= cos(\theta)sin(\phi)\\
	y &= sin(\theta)sin(\phi)\\
	z &= cos(\phi)
\end{align}
$x$, $y$ and $z$ are the corresponding cartesian coordinates.
\vspace{5mm}


\begin{figure}
     \centering
     \begin{subfigure}[b]{0.45\textwidth}
         \centering
         \includegraphics[width=\textwidth]{pics/disc5}
         \caption{In steps of 5 discretization with 1296 points}
         \label{fig_5step}
     \end{subfigure}
     \hfill
     \begin{subfigure}[b]{0.45\textwidth}
         \centering
         \includegraphics[width=\textwidth]{pics/fibolat1}
         \caption{Fibonacci lattice-based discretization with 756 points.}
         \label{fig_fibolat}
     \end{subfigure}
     \hfill
     
\caption{Comparison of two different discretization patterns. Fibonacci lattice based discretization on right shows a more even distribution of points than the latitude-longitude lattice. The minimum density is approximately the same in both graphs despite the difference in point counts.}
     \label{fig_5stepfibolat}
\end{figure}

\newpage
\subsection{Importance of lattice density}\label{ss_lattice_density}
Using the correct lattice density is important for using exhaustive search algorithms for panel angle estimation. Regardless of the lattice point count, a discrete lattice is unlikely to ever include the best fit in the whole $\mathbb{R}^2$ parameter subspace. This means that if the lattice density is low, distance between best fit and closest lattice points can result in significant errors. However as incresing the lattice density increases the computational cost of angle estimation, choosing a good density is an optimization problem.

With \textit{in-steps-of-n} lattices, the main benefit is easy readibility. If the lattice is given one point for latitudinal and longitudinal degree, resulting in $360*90=32400$ points, then the lattice can be aligned so that each tilt and azimuth degree pair where angles are integers is tested. This discretization makes the results easily understanble as if the known installation angles are given as integers, a point representing the exact known installation angles already exists on the lattice. The same applies for lower density in \textit{in-steps-of-n} lattices.

If a Fibonacci lattice or some other discretization method is used, evaluating the performance of the fitting algorithm is not as easy. With an $n$ point Fibonacci lattice the distance between lattice points varies slightly and and it may be difficult to estimate where the closest points to the known installation angles are.




%The density of lattices is an important measurable charasteristic which proves useful during the optimization of angle estimation algorithm. The most useful metric would be the sphere center angle distance between neighboring points. This would be useful as it can be used for determining whether errors in predictions are lattice or function fitting related. For example, if the lattice neighbors are approximately 1 degree away from oneanother and the predicted angle is 5 degrees off from the known installation angle, then the error is caused by model fitting and not lattice density as there must have been multiple lattice points closer to the known angle point than the discovered best fit. However if grid density is near to or lower than angle estimation error, the lattice is likely to be a contriburing to angle estimation errors.

%Calculating neighborign center angle distances for both \textit{in-steps-of-n} and Fibonacci lattices is somewhat challenging. In \textit{in-steps-of-n}, the value \textit{n} can be used as an estimate for max center angle distance as \textit{n} will always be the tilt distance to the nearest neighbor with different tilt angle. With Fibonacci lattices the easiest way of estimating center angle distances is taking the coordinates of the first two lattice points and calculating their center angle distance with earlier error equation \ref{errorangle}. These first two points should be used as Fibonacci lattice points are distributed on a single arm spiral pattern, resulting in later sequential points being further from one another.

%Another method for calculating Fibonacci lattice point distances is dividing the surface area of the angle space by the amount of calculated lattice points. This area-per-point value could then be used in order to estimate how far points are from oneanother on average. In later sections, the first two points derived distance will be used.

\newpage
\section{Solving panel angles}
Now that the geographic location and multiplier value of installation are known to be solvable and fitness functions have been defined, the next step is solving the panel angles. The simple method is evaluating all points on a sufficiently dense lattice and choosing the point with the lowest delta value.

Figure \ref{fig_polar10} contains 10 fibonacci lattice points and their normalized delta values. The best fit was at tilt 31.79$^\circ$, azimuth 153.79$^\circ$ and delta value of 369W. The lattice density leaves large gaps between lattice points and this found best fit is the closest point to the known installation angles of 15$^\circ$ and 135$^\circ$. Center angle error as per \ref{errorangle} is 18.17$^\circ$.




%is to solve the panel installation angles. The chosen method relies on splitting angle space into $n$ discrete points and evaluatin each of their fitness by calculating an error value. Here the angle space was split into 10 discrete points with fibonacci lattice \ref{angle_space_discretization} and the fitness of each point was evaluated with area error \ref{areaerror} and multiplier matching \ref{function_multipliermatch} functions. %The tilt-azimuth pair with the lowest error value was 31 


\begin{figure}[h]
\begin{floatrow}
\ffigbox{%
  \includegraphics[width=1\linewidth]{pics/halfdome} % WAS pics/10p_fibo_fit_text
}{
  \caption{Polar plot of test points for a single day of data from FMI Kumpula dataset.}
  \label{fig_polar10}
}
\capbtabbox{%
  \begin{tabular}{c|c|c} \hline

Tilt$^\circ$ & Azimuth$^\circ$ & Delta(W)\\
\hline
18.19 & 291.25 & 3128\\
\textbf{31.79} & \textbf{153.74} & \textbf{369}\\
41.41 & 16.23 & 3873\\
49.46 & 238.72 & 4183\\
56.63 & 101.22 & 3356\\
63.26 & 323.71 & 6888\\
69.51 & 186.2 & 1759\\
75.52 & 48.69 & 5814\\
81.37 & 271.18 &  6626\\
87.13 & 133.68 & 2938\\
\hline\hline
\end{tabular}
}{%
  \caption{Tilt, azimuth and error table for single day.}
}
\end{floatrow}
\end{figure}

The fit achieved in \ref{fig_polar10} is not very good and better results can be achieved by increasing the lattice density. The trivial method is to use a fibonnacci lattice with a higher point count, for example the fit achieved with a 10 000 point lattice \ref{10k_fits_new_helsinki} found the best fit at tilt 14.79$^\circ$, azimuth 136.2$^\circ$ , delta value of 136.2W and center angle error of 0.3659 degrees$^\circ$. This is a much better fit but increasing the lattice density comes with a higher computational cost. Depending on code optimizations, evaluating a single day from the dataset against a 10 000 point lattice can from a minute up to several hours.

\begin{figure}[h]
\centering
\includegraphics[width=0.8\linewidth]{pics/10k_new_model} % WAS pics/measured_vs_simulated
\figcaption{Results of a 10 000 datapoint lattice fitting against a single day from FMI Helsinki dataset. Center angle error between the found best fit and known installation angles is 0.3659 degrees$^\circ$.}
\label{10k_fits_new_helsinki}
\end{figure}



%# Best fit was tilt: 14.79 azimuth: 136.2 fitness:133.3 10k lattice result !






%\noindent Now that the method can be seen to work, it is time to improve the results. This can be done by generating larger lattices and thus by evaluating higher amount of datapoints, the algorithm has a higher chance of finding the global minimum error point. As per \ref{ss_lattice_density}, Fibonacci lattice of 1000 points would have the angular resolution of approximately 4 degrees where as 10000 points would be near to 1.5 degrees. The performance can also be improved by evaluating best fits for multiple days at once. The resulting point cloud of best fits can then be used for averaging out noise in the predictions.

%The plot \ref{fig_polar_multiyear} is a result of using the angle finding algorithm on 41 days from FMI Kumpula dataset with a Fibonacci lattice of 1000 points. The two darker spots near the center of the graph are the two most common best fits, [17.0$^\circ$, 138.4$^\circ$] with 17 and [22.7$^\circ$, 143.1$^\circ$] with 16 out of 41 days. These groupings are as close to the known installation angles of [15$^\circ$,  135$^\circ$] as could be expected from a 1000 point lattice. The next step is tightening the cluster, this can be done by adjusting the smoothness requirement of the cloud free day algorithm or by restricting the day range. In \ref{fig_polar_multiyear_summer} the tightening was accomplished with day range restrictions and 22 days were accepted by the algorithm.


%In \ref{fig_polarplot_13days} the polar plot is a result of taking the 13 best days from FMI Kumpula dataset year 2018 and using a fibonacci lattice with 500 test points. The most common best fit was [27.5$^\circ$, 150.8$^\circ$] with 5 occurances followed by [24.1$^\circ$, 138.4$^\circ$] with 4 days. These values are close to the 





%As all of the estimates converge on two neighboring points, the final step is increasing the lattice resolution further. With 10 000 point lattice, the cluster tightened further \ref{fig_polar_multiyear_helsinki10k}. Out of the 22 days, 7/22 or 32\% had best fit at [19.34, 136.87] with error of 4.38 degrees and 6/22 or 27\% at [17.74, 135.06] with error of 2.74 degrees. Rest of the best fits were distributed in smaller clusters near these points. The lowest angle distance best fit was [17.09, 130.32] with angle distance of 2.46 degrees. As the angle distance errors are higher than the discretization resolution and as the predicted angles are systematically biased, it would seem that the error is caused by the solar irradiance model or model fitting and not angle space discretization.


%As the angle distance errors are higher than the discretization resolution and as the results seem biased, the errors of the estimation algorithm would seem to be caused by inaccuracies in model fitting and not angle space discretization.




%\begin{figure}[h]
%\centering
%\includegraphics[width=0.8\linewidth]{pics/10kfitshelsinkiplot}
%\caption{Comparison between measured values, simulated values with known installation parameters and best found fit.}
%\label{fig_10kfitshelsinkiplot}
%\end{figure}



%day 141 predicted 17.74 135.06 delta degrees: 2.74 x
%day 145 predicted 19.92 141.6 delta degrees: 5.3
%day 148 predicted 19.34 136.87 delta degrees: 4.38 	x
%day 179 predicted 17.74 135.06 delta degrees: 2.74 x
%day 197 predicted 21.37 143.41 delta degrees: 6.87 y
%day 198 predicted 18.37 139.79 delta degrees: 3.64
%day 199 predicted 19.34 136.87 delta degrees: 4.38		x
%day 157 predicted 19.92 141.6 delta degrees: 5.3
%day 169 predicted 18.37 139.79 delta degrees: 3.64
%day 202 predicted 17.74 135.06 delta degrees: 2.74 x
%day 208 predicted 21.37 143.41 delta degrees: 6.87 y
%day 143 predicted 19.34 136.87 delta degrees: 4.38		x
%day 155 predicted 19.92 141.6 delta degrees: 5.3
%day 164 predicted 19.92 141.6 delta degrees: 5.3
%day 166 predicted 17.74 135.06 delta degrees: 2.74 x
%day 171 predicted 19.34 136.87 delta degrees: 4.38		x
%day 174 predicted 17.09 130.32 delta degrees: 2.46
%day 175 predicted 17.74 135.06 delta degrees: 2.74 x
%day 177 predicted 17.74 135.06 delta degrees: 2.74 x
%day 178 predicted 19.34 136.87 delta degrees: 4.38		x
%day 179 predicted 19.34 136.87 delta degrees: 4.38		x
%day 200 predicted 19.34 136.87 delta degrees: 4.38		x



%day 145 predicted 32.09 196.89 delta degrees: 18.65 x
%day 192 predicted 30.5 198.01 delta degrees: 16.96 y
%day 199 predicted 30.5 198.01 delta degrees: 16.96 y
%day 210 predicted 34.52 197.58 delta degrees: 20.9 z
%day 212 predicted 32.09 196.89 delta degrees: 18.65 x
%day 213 predicted 35.07 194.65 delta degrees: 21.86 k
%day 167 predicted 27.07 200.24 delta degrees: 13.37 i
%day 143 predicted 30.5 198.01 delta degrees: 16.96 y
%day 144 predicted 28.83 199.13 delta degrees: 15.21 l
%day 145 predicted 28.83 199.13 delta degrees: 15.21 l
%day 162 predicted 27.07 200.24 delta degrees: 13.37 i 
%day 167 predicted 27.07 200.24 delta degrees: 13.37 i 


\newpage
\subsection{Evaluation of exhaustive search results}
The estimated installation angles installations are fairly good. A delta of less than $4.5^\circ$ as was achieved with FMI Kumpula is small enough to be a result of measurement or rounding error. The estimates for FMI Kuopio are off by more than 10 degrees which is less encouraging. As the reported angles for FMI Kuopio were $15^\circ$ and $217^\circ$, it would seem like the angle measurements were rounded to nearest degree. This would eliminate the reporting accuracy as a plausible cause for the estimation errors and thus either there has been a reporting error or that the installation angle estimation algorithms are not performing as well for FMI Kuopio dataset.


\begin{figure}[h]
\begin{floatrow}
\capbtabbox{%
  \begin{tabular}{r|c|c|c} \hline
\multicolumn{4}{c}{FMI Kumpula}\\
\hline
n & Tilt & Azimuth & Error\\
\hline
7 & $19.34^\circ$ & $136.87^\circ$ & $4.38^\circ$\\
6 & $17.74^\circ$ & $135.06^\circ$ & $2.74^\circ$\\
4 & $19.92^\circ$ & $141.60^\circ$ & $5.30^\circ$\\
2 & $21.37^\circ$ & $143.41^\circ$ & $6.87^\circ$\\
1 & $21.37^\circ$ & $143.41^\circ$ & $6.87^\circ$\\
1 & $17.09^\circ$ & $130.32^\circ$ & $2.46^\circ$\\

\hline\hline
\end{tabular}
}{%
  \caption{Estimation results table for FMI Kumpula.}
}
\capbtabbox{%
  \begin{tabular}{c|c|c|c} \hline
\multicolumn{4}{c}{FMI Kuopio}\\
\hline
n & Tilt & Azimuth & Error\\
\hline
3 & $27.07^\circ$ & $200.24^\circ$ & $13.37^\circ$\\
3 & $30.50^\circ$ & $198.01^\circ$ & $16.96^\circ$\\
2 & $28.83^\circ$ & $199.13^\circ$ & $15.21^\circ$\\
2 & $32.09^\circ$ & $196.89^\circ$ & $18.65^\circ$\\
1 & $34.52^\circ$ & $197.58^\circ$ & $20.90^\circ$\\
1 & $34.52^\circ$ & $197.58^\circ$ & $20.90^\circ$\\
1 & $35.07^\circ$ & $194.65^\circ$ & $21.86^\circ$\\


\hline\hline
\end{tabular}
}{%
  \caption{Estimation results table for FMI Kuopio.}
}
\end{floatrow}
\end{figure}


Figures \ref{10khelsinkiplot} and \ref{10kkuopioplot} shows that the models based on best found fits are better fits than simulations done with the known parameters. This is true for both Helsinki and Kuopio datasets. This would suggest that the model fitting works as intended and that either the model is inaccurate or there is an error in reported panel angles. The more likely cause of the two is the the solar irradiance model and for some undetermined reason the error of the model is more significant for the Kuopio installation. Possible causes could be related to lower sun angles resulting in higher reflective losses or shadowing during last production hours which would also explain the uneven structure visible in the last non-zero hours in \ref{10kkuopioplot}.





\section{Solving panel angles iteratively}
The exhaustive search used in earlier section suggest that the method is capable of estimating panel installation angles accurately. However evaluating 10 000 point lattices is somewhat inelegant and this can be avoided by using multiple less dense lattices iterarively if the fitness space satisfies a couple of requirements.

The first is that the fitness space should be smooth. This smoothness doesn't have to be perfect, fine patterns and details in the fitness space surface do not cause issues with iterative search algorithms if their size is small enough not to be observable to the iterative lattices. Anothe requirement is that the fitness base should contain as few basin or bowl sections as possible with the best fit being the minimum point of the largest and deepest basin.

The exhaustive search visualizations show that the gradient visible in the fitness space is smooth enough not to result in visible noise patterns in the figure \ref{10k_fits_new_helsinki}. This may not be true for all cases but as long as the selected clear cloud free day and the simulation do not both contain a fine 



 In computer science the following method would perhaps be classified as a divide-and-conquer algorithm where as mathematicians would call it gradient search.

\noindent \textbf{Iterative panel angle estimation algorithm}
\begin{enumerate}
	\item Choose a cloud free day from the dataset for evaluation.
  \item Choose a starting or "center" point from the angle space. This can be either the best result from a low density lattice or a fixed point such as tilt 45$^\circ$ azimuth 180$^\circ$.
  \item Evaluate the fitness at the center point and store that as the center point fitness.
  \item Choose a few points near this center point within a given distance and evaluate their fitness. If any of the neighboring points results in a better fit than the center point, this point will then be chosen as the new center point.
  \item Repeat steps 3 and 4 until step 4 does not find a better fit. When this happens, decrease the distance used for the local lattice in step 4.
  \item Repeat steps 3,4 and 5 for a set number of times. Last center point is the best iteratively found fit.
\end{enumerate}


\noindent Step 4 is the first non-trivial step in the algorithm. The selection of neighboring points can be done in multiple ways, the cross pattern search method shown in figure \ref{fig_iterative_visual_1} is based first transforming the starting angle tilt 45$^\circ$  and azimuth 270$^\circ$ to a cartesian coordinate plane point and the generatin 4 nearby cartesian plane points at distance of 0.3 which are then transformed back into angle space. The equations used for space transformations are \ref{angle_space_to_unit_circle} and \ref{unit_circle_to_angle_space}. This space transformation method results in visually pleasing plots and the cartesian distance of 0.3 is easily adjustable in step 5 of algorithm.

\newpage

\noindent \textbf{Unit circle to angle space equations}

\begin{equation}
\begin{split}
\label{unit_circle_to_angle_space}
d &= x^2+y^2\\
Tilt &= \sqrt{d}*90^\circ\\
Azimuth &= tan^{-1}2(x/d, y/d)
\end{split}
\end{equation}

\noindent \textbf{Angle space to unit circle equations}
\begin{equation}
\begin{split}
\label{angle_space_to_unit_circle}
d &= Tilt/90^\circ \\
x &= cos(Azimuth)d \\
y &= sin(Azimuth)d \\
\end{split}
\end{equation}

\vspace{5mm}





\begin{figure}
     \centering
     \begin{subfigure}[b]{0.45\textwidth}
         \centering
         \includegraphics[width=\textwidth]{pics/iterative_1_step}
         \caption{Iterative best fit search after 1 cross pattern search.}
         \label{fig_iterative_1_step}
     \end{subfigure}
     \hfill
     \begin{subfigure}[b]{0.45\textwidth}
         \centering
         \includegraphics[width=\textwidth]{pics/iterative_6_step}
         \caption{Iterative best fit search after 6 cross pattern searches.}
         \label{fig_iterative_6_step}
     \end{subfigure}
     \hfill
     
\caption{Visualization of iterative best fit search algorithm. Numbers next to markers indicate that a point was the best found during n:th cross pattern search. Number 5 is missing from \ref{fig_iterative_6_step} as no new best fit was found, resulting in search distance deacrease. Red circle marks the best found fit.}
     \label{fig_iterative_visual_1}
\end{figure}









\chapter{Conclusion}
PV system parameter estimation results for the FMI Helsinki dataset are very good to excellent. Angle estimation results with a center angle delta of less than 1$^\circ$ were achievable with both iterative and exhaustive algorithms. Geolocation estimation proved to be more difficult with the scatter pattern from multiple years having small amount of bias and a fair amount of noise. Scatter formation is spread around the FMI Helsinki installation and is approximately 50km by 200km in dimensions. Due to small sample size of 5 datapoints this algorithm is harder to evaluate.

Results for FMI Kuopio dataset were noticeably worse. Center angle delta with panel installation angles was approximately 13$^\circ$ regardless of estimation method used. Similarly the scatter pattern in geolocation estimation resulted in a 50km by 300km region with outliers.

The differences in the algorithm performances between the datasets can partially be explained by the noise present in the Kuopio data where something would appear to be casting shadowns onto the panels. Differences may also be partially caused by parameter estimation algorithm parameters such as used day ranges which affect the results of estimation algorithms.




%The initial goal of this thesis was to find a simple way of estimating parameters of solar PV installations and this goal has been accomplished with moderate success. Some of the algorithms are thousands of lines of long, but the underlying mathematics was still kept relatively simple. As a result, the code can be understood and modified by a wider audience.

%From the perspectives of mathematics and programming, model fitting problems are not particularly difficult. In this thesis, the challenges rose from optimization, understanding patterns in the data and discovering where the limits of the estimation algorithms come from. The insights gained while tackling these issues may be more valuable in to other researchers than the final estimation algorithms.





%The most significant of which are the center angle error function, Fibonacci-lattice based angle space discretization and angle space resolution estimates. These were not mentioned in the cited literature and while the are most likely already used in other fields, they would most likely prove to be useful for similar studies conducted in the future.





%The initial goal of this thesis was to find a simple way of estimating parameters of solar PV installations and this goal has been accomplished with moderate success. Some of the algorithms are thousands of lines of long, but the underlying mathematics was still kept simple. As a result, the code can be understood and modified by a wider audience. This is in particular contrast with AI and machine learning based approaches which often provide good results but which tend to be less insightful.

%From the perspectives of mathematics and programming, model fitting problems are not particularly difficult. In this thesis, the challenges rose from optimization, understanding patterns in the data and discovering where the limits of the estimation algorithms come from. The insights gained while tackling these issues may be more valuable to other researchers than the final estimation algorithms. The most significant of which are the center angle error function, Fibonacci-lattice based angle space discretization and angle space resolution estimates. These were not mentioned in the cited literature and while the are most likely already used in other fields, they would most likely prove to be useful for similar studies conducted in the future.

%While experimenting with the datasets, some interesting traits and phenomena were observed, some of which could warrant their own studies. For example, the figure \ref{10kkuopioplot} shows that the last hours of the specific day are noisy. This noise may play a significant role in the prediction erros and thus further studies in detection and classification of noise types in solar PV datsets would prove to be useful for all who perform analysis on solar PV installations. Perhaps the largest apparent obstacle in noise detection and classification studies is the temporal resolution as noise profiles of clouds, shadowing structures or temperature fluctuations may prove to be impossible to detect accurately at low temporal resolutions.

%Lastly I would like to encourage other researchers to publish their research and code openly. During preminary research and literature reviews, the code examples or datasets which were used for research papers did not seem to be available. While research code may not be useful as is, 


% open research should be encouraged. During preminary research and literature reviews, there did not seem to be 

%Lastly the state of open source research is somewhat concerning. Solar energy plays a significant role in green energy transition and 

\appendix
\section{Source code}
In total, the project source consists of ~3000 lines of source code divided into multiple files and directories for easier usage. Including the full source in the appendix would not be feasible but the source is available online at github\cite{project_source}. These project files can be downloaded and tested in python environments which meet the recommended python and python package versions detailed in the included Readme.md -file. This same file also includes further description on project structure and usage.


\section{Data files}
The solar PV datasets used in this thesis are private but available upon request. PV datasets are planned to be released along with a data paper at a later point.

The program expects FMI Helsinki and Kuopio csv -files to have names "fmi-helsinki-2021.csv" and "fmi-kuopio-2021.csv" respectively. The exact filenames are not important and files can either be renamed to match these patterns or the data loading code in "solar\_power\_data\_loader2.py" can be modified to match the used .csv files.

Datasets from other sources can also be used but their usage will require some modifications to the program code. If the temporal resolution or csv structure are different from the structure detailed in table\ref{table_fmi_kumpula_csv}, these modifications may be significant.
%

\renewcommand{\bibname}{References}  %  change References to Viitteet if writing in Finnish
%
\phantomsection
%
\addcontentsline{toc}{chapter}{\bibname}
\renewcommand{\baselinestretch}{1}
\label{bibbib}
\bibliography{example_refs.bib}
%
%\include{thesis_app} % If there are no appendices, remove this line.
%




\end{document}
