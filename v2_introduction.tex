\chapter{Introduction}

This thesis examines a specific applied mathematics problem suggested by the Finnish Meteorological institute FMI. The goal is to determine the geographic location and panel installation angles of photovoltaic solar power installations using only the power output data. The chosen method breaks down the problem of solving the geographic location and panel angles into separate algorithms. In practice, this means that the algorithms used can be less complex as they do not have to solve every unknown variable simultaneously, and visualizations of individual algorithms should also be more straightforward. Using multiple algorithms also splits the parameter space into smaller spaces, thus improving the performance of fitting algorithms. And the final benefit is the ability to focus on solving only the unknown parameters. For example, if the geolocation of a system is known and the panel installation angles are not, instead of estimating the geolocation, the known geolocation can be used for panel angle estimation, resulting in higher prediction accuracy.


The applications of parameter estimation algorithms would be in improving the quality of metadata in solar PV datasets. This could have implications for solar PV research, but the existence of such algorithms poses privacy and security-related questions as well. Whether these research benefits and concerns are realized depends on the accuracy, and to some extent, the ease of use of the algorithms.



A similar study was done by N. Haghdadi et al. in 2017\cite{navid_australian_article}. The 2017 article contains results from five case studies where the standard deviation of longitude prediction errors is less than $1.5^\circ$, reaching as low as $0.08^\circ$ with case study 1-2. The standard deviation of latitude predictions is higher at less than 3.5$^\circ$ with the best result in case study 2-2 with a standard deviation of 1.65$^\circ$. Azimuth and tilt predictions are reported as two separate angle error values with azimuth prediction errors reaching values between 4$^\circ$ and 27$^\circ$ and tilt 1.3$^\circ$ to 11.5$^\circ$. These forementioned results are not directly comparable to the results shown in this thesis due to different datasets, geographic location and local climate, but they provide some perspective.



Another article of relevance written by M.K. Williams et al. in 2012\cite{older_solar_solver_article} proposes multiple methods for determining the locations and orientations of solar PV installations. Perhaps the most interesting contribution of the 2012 article is the network approach to determining geographic location. This method relies on a grid of installations with known and accurate geolocation and installation angles. According to the authors, this networked approach works up to a 10-mile accuracy when the grid of known installations is dense around the estimated installation. This could make the network approach preferable for electric companies or institutions with large amounts of data. But as of now, it does not seem usable for FMI.
