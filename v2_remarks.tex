\chapter{Conclusion}
PV system parameter estimation results for the FMI Helsinki dataset are very good to excellent. Angle estimation results with a center angle delta of less than 1$^\circ$ were achievable with both iterative and exhaustive algorithms. Geolocation estimation proved to be more difficult with the scatter pattern from multiple years having small amount of bias and a fair amount of noise. Scatter formation is spread around the FMI Helsinki installation and is approximately 50km by 200km in dimensions. Due to small sample size of 5 datapoints this algorithm is harder to evaluate.

Results for FMI Kuopio dataset were noticeably worse. Center angle delta with panel installation angles was approximately 13$^\circ$ regardless of estimation method used. Similarly the scatter pattern in geolocation estimation resulted in a 50km by 300km region with outliers.

The differences in the algorithm performances between the datasets can partially be explained by the noise present in the Kuopio data where something would appear to be casting shadowns onto the panels. Differences may also be partially caused by parameter estimation algorithm parameters such as used day ranges which affect the results of estimation algorithms.




%The initial goal of this thesis was to find a simple way of estimating parameters of solar PV installations and this goal has been accomplished with moderate success. Some of the algorithms are thousands of lines of long, but the underlying mathematics was still kept relatively simple. As a result, the code can be understood and modified by a wider audience.

%From the perspectives of mathematics and programming, model fitting problems are not particularly difficult. In this thesis, the challenges rose from optimization, understanding patterns in the data and discovering where the limits of the estimation algorithms come from. The insights gained while tackling these issues may be more valuable in to other researchers than the final estimation algorithms.





%The most significant of which are the center angle error function, Fibonacci-lattice based angle space discretization and angle space resolution estimates. These were not mentioned in the cited literature and while the are most likely already used in other fields, they would most likely prove to be useful for similar studies conducted in the future.





%The initial goal of this thesis was to find a simple way of estimating parameters of solar PV installations and this goal has been accomplished with moderate success. Some of the algorithms are thousands of lines of long, but the underlying mathematics was still kept simple. As a result, the code can be understood and modified by a wider audience. This is in particular contrast with AI and machine learning based approaches which often provide good results but which tend to be less insightful.

%From the perspectives of mathematics and programming, model fitting problems are not particularly difficult. In this thesis, the challenges rose from optimization, understanding patterns in the data and discovering where the limits of the estimation algorithms come from. The insights gained while tackling these issues may be more valuable to other researchers than the final estimation algorithms. The most significant of which are the center angle error function, Fibonacci-lattice based angle space discretization and angle space resolution estimates. These were not mentioned in the cited literature and while the are most likely already used in other fields, they would most likely prove to be useful for similar studies conducted in the future.

%While experimenting with the datasets, some interesting traits and phenomena were observed, some of which could warrant their own studies. For example, the figure \ref{10kkuopioplot} shows that the last hours of the specific day are noisy. This noise may play a significant role in the prediction erros and thus further studies in detection and classification of noise types in solar PV datsets would prove to be useful for all who perform analysis on solar PV installations. Perhaps the largest apparent obstacle in noise detection and classification studies is the temporal resolution as noise profiles of clouds, shadowing structures or temperature fluctuations may prove to be impossible to detect accurately at low temporal resolutions.

%Lastly I would like to encourage other researchers to publish their research and code openly. During preminary research and literature reviews, the code examples or datasets which were used for research papers did not seem to be available. While research code may not be useful as is, 


% open research should be encouraged. During preminary research and literature reviews, there did not seem to be 

%Lastly the state of open source research is somewhat concerning. Solar energy plays a significant role in green energy transition and 
